\documentclass[UKenglish]{ifimaster}  %% ... or USenglish or norsk
\usepackage[utf8]{inputenc}           %% ... or latin1
\usepackage[T1]{fontenc,url}
\urlstyle{sf}
\usepackage{babel,textcomp,csquotes,duomasterforside,varioref,graphicx}
\usepackage[backend=biber,style=numeric-comp, sorting=none]{biblatex}
\usepackage{amsmath,amssymb,amsfonts}
\usepackage{algorithmic}
\usepackage{textcomp}
\usepackage{xcolor}
\usepackage{textgreek}
\usepackage{float}
\usepackage{graphicx}
\usepackage{chemformula}
\usepackage{cleveref}
\usepackage{caption}
\usepackage{subcaption}
\usepackage{booktabs}
\usepackage{verbatim}
\usepackage{chemformula}
\usepackage{multirow}
\usepackage{comment}



\raggedbottom %%reduces the gaps between the paragraphs

\title{Thesis Title}        %% ... or whatever
\subtitle{A subtitle of your thesis }         %% ... if any
\author{Author name}                      %% ... or whoever 

\addbibresource{bibliography.bib}  
\graphicspath{{figures/}}
\DeclareUnicodeCharacter{2212}{-}

\includeonly
{
	%chapter/chapter_background,
   	%chapter/chapter_2,
    %chapter/chapter_3,
    %chapter/chapter_4,
    chapter/chapter_5,
    %chapter/chapter_6,
    %chapter/chapter_7,
    %chapter/chapter_8,
    %chapter/chapter_9,
    %chapter/chapter_11,
    %chapter/chapter_12,
    %chapter/chapter_13
}


\begin{document}
\duoforside[dept={Department Name <change at main.tex>},   %% ... or your department
  program={Master's Program Name <change at main.tex>},  %% ... or your programme
  long]                                        %% ... or long

\frontmatter{}  
\chapter*{Abstract}                   %% ... or Sammendrag or Samandrag

\tableofcontents{}
\listoffigures{}
\listoftables{}

\chapter*{Preface}                    %% ... or Forord

\mainmatter{}

\chapter{Introduction}                  %% ... or Bakgrunn

\textbf{some introduction on the importance of discovering new materials and alloying}.\\
\textbf{Need something on thermoelectricity related to both the band gap and high-entropy alloys.}

High-entropy alloys is a novel class of materials based on alloying multiple components, as opposed to the more traditional binary alloys. This results in an unprecedented opportunity for discovery of new materials with a superior degree of tuning for specific properties and applications. Recent research on high-entropy alloys have resulted in materials with exceedingly strong mechanical properties such as strength, corrosion and temperature resistance, etc \textbf{find references}. Meanwhile, the functional properties of high-entropy alloys is vastly unexplored. In this study, we attempt to broaden the knowledge of this field, the precise formulation of this thesis would be an exploration on the possibilities of semiconducting high-entropy alloys. 

A key motivation of this thesis is the ability to perform such a broad study of complex materials in light of the advances in material informatics and computational methods. In this project, we will employ Ab initio methods backed by density functional theory on top-of the line supercomputers and software. 20 years ago, at the breaking point of these methods, this study would have been significantly narrower and less detailed firstly, but secondly would have totaled ... amount of CPU hours to complete (\textbf{Calculate this number}). In the addition to the development in computational power, is also the progress of modeling materials, specifically we will apply a method called Special Quasi-random Structures (SQS) to model high-entropy alloys or generally computationally complex structures. Together with the open landscape of high-entropy alloys described above, these factors produce a relevant study in the direction of applying modern computational methods to progress the research of a novel material class and point to promising directions for future research.   
 
In specifics, this thesis revolve around the electrical properties of high-entropy alloys, mainly the band gap as this is the key indicator for a semiconducting material and it's applicability. Semiconductors are the building blocks in many different applications in today's world, ranging from  optical and electrical devices, to renewable energy sources such as solar and thermoelectricity. Given the economic and sustainable factors concerning silicon, in addition to its role in relevant applications such as microelectronics and solar power. Silicon emerges as a natural selection to build our alloys around. Furthermore, the development and research on both high entropy alloys and metal silcides have been heavily centered around 3d transition metals. Keeping in line with the economic and environmental factors, we will continue this direction by focusing on high entropy stabilized sustainable and economic 3d metal silicides \textbf{Not happy with this writing}. Throughout the study we will analyze a great number of permutations of 3d silicides, from different initial metal silicides such as $CrSi_2, FeSi_2, MnSi_{1.75}, Fe_2Si$, each with distinct properties relating to the band gap, crystal structure and metal to silicon ratio. In addition, the permutations include numerous metal distributions and elements within the 3d-group of metals. Examples are Co, Cr, Fe, Mn, and Ni. 

Given a background in high-entropy alloys, one could ask if this study is truly sensible. In the later sections we will cover the details of this field, and it quickly become clear that the materials investigated in this study does not fall under the precise definition of high-entropy alloys, nor do we intend to explore the properties and factors relating to high-entropy stabilized alloys such as the configurational entropy, phase stability and finite temperature studies. However this study is motivated from the discovery of these materials and promising properties, and venture into a more hypothetical space of materials, enabled by the computational methods available to study the potential properties of such materials. On the other hand, very recent studies \textbf{Mari, and other HEA silicide study} have experimentally synthesized high-entropy disilicides, thus in some way justifying the direction of this project. 

We begin this project by reviewing key concepts of solid-state physics for readers lacking a background in materials science, and an introduction to the base 3d silicides of the experimental work. Later follows a theoretic walk-through of the relevant concepts of this thesis, these topics include high-entropy alloys, special quasi-random structures, and density functional theory. Next we shine light on the implementation of DFT in this project, and other computational details required to reproduce the results in this thesis, such as the use of the Vienna Ab Initio Simulation Package (VASP) and implementation of SQS. Finally we present the results of our study, these include the band gap and electronic properties of various structures and the success and challenges of the computational methods applied throughout the study. 

\part{Theory}                    %% ... or ??
\chapter{High-Entropy alloys}
\label{sec:HEA}

High-Entropy Alloys (HEAs) has become an increasingly popular field in materials science, from to the large flexibility and possibilities for discovering new materials with unique properties. Since the original discovery in 2004, as of 2015 there has been over 1000 published journal articles on high-entropy alloys \cite{hea2016_ch1}. In the following section we will cover the fundamentals and some applications of high-entropy alloys. This section is based on the fantastic description of HEAs in the book "High-Entropy Alloys - Fundamentals and Application", in particular chapters 1,2,3 and 7 \cite{hea2016_ch1}, \cite{hea2016_ch2}, \cite{hea2016_ch3}, \cite{hea2016_ch7}, and references therein.

\section{Fundamentals} 
A high entropy alloy can in a way be compared to a smoothie. In a smoothie one can produce unique combinations of flavors and nutritional values based on both the properties of the individual fruits and vegetables, and their interplay in the mixture. In materials science, a similar approach can be applied to generate a large range of materials with tunable properties depending on the intended application. In respect to HEAs, examples can be increased strength, ductility, corrosive resistance and low thermal conductivity. Moving on from the rather banal fruit analogy, a high-entropy alloy can be defined from two conditions:
\begin{enumerate}
    \item The material consists of at least 5 distinct elements, where each element contribute between 5-35$\%$ of the composition.
    \item The total configurational entropy is greater than 1.5R, where R is the gas constant. 
\end{enumerate}
The latter is an especial case for high-entropy alloys. The ideal configurational entropy of a random N-component solid-solution is described as
\begin{equation}
\Delta S_{\text{config}} = -R \sum_{i=1}^{N}X_i\ln X_i,
\end{equation}
where $X_i$ is the mole fraction of the $i$th component. Its clear that $\Delta S_{\text{config}}$ increase with a higher number of constituents in the mix. For instance, the ideal configurational entropy of a equimolar binary alloy is 0.69 R, as opposed to 1.61 R in a five-component equimolar alloy. If we neglect the other factors that influence the formation of solid solutions (will be covered later), from Gibbs free energy
\begin{equation}
\Delta G_{\text{mix}} = \Delta H_{\text{mix}} - T\Delta S_{\text{mix}},
\end{equation} 
the two primary factors in formation of solid solution are the mixing enthalpy, which is the driving force to form compounds, and the mixing entropy which is the driving force to form random solid solutions. At elevated temperatures especially, the energy associated to the entropy of the system becomes comparative to the mixing enthalpy and can impact the formation. In summary, the overall concept of high-entropy alloys is that through alloying a greater number of elements, the gain in configurational entropy of the system prohibits the formation of intermetallic compounds, in favor of random solid solution. The random term simply relate to the various components occupying lattice positions based on probability. In fact, a narrower definition of high-entropy alloys would be structures with a single-phase disordered solid solution. The two "definitions" given previously, can be considered as guidelines for the latter.
 
Although the mixing entropy mentioned above plays a central role in the formation, there are other factors to consider that can either favor or oppose the formation of a single disordered phase. One of these is the atomic size effect, which is related to the differences in atomic size between constituents in the alloy. This quantity is denoted $\delta$. Y. Zhang et al. in 2008 illustrated the relationship between $\Delta H_\text{mix}$ and $\delta$. When $\delta$ is very small, in other words when the alloys are comprised of elements with similar atomic sizes, the elements have an equal probability to occupy lattice sites to form solid solutions, but the mixing enthalpy is not negative enough to promote formation of solid solution. Increasing $\delta$ results in greater $\Delta H_\text{mix}$, but leads to a higher degree of ordering. The formation of solid solution high-entropy alloys occur in a narrow range of $\delta$ values, that satisfy both the enthalpy of mixing and the disordered state. Recently, Yang and Zhang proposed the parameter $\Omega$ to evaluate the stability of high-entropy alloys. This quantity is a product of the melting temperature $T_m$, mixing entropy and mixing enthalpy in the following relation

\begin{equation}
\Omega = \frac{T_\text{m} \delta S_\text{mix}}{|\Delta H_\text{mix}|}.
\end{equation}

They found that the formation of single disordered solid solution occurs at $\Omega \geq 1.1$ and $\delta \leq 6.6\%$, while compounds such as intermetallics form for greater values of $\delta$ and lesser values of $\Omega$. Similarly, replacing the atomic size effect with the number of elements results in an equivalent condition. These findings are summarized in figure 2.1.
\begin{figure}[H]
\centering
\begin{subfigure}{.85\textwidth}
\includegraphics[width=\textwidth]{theory/heaformation1.jpeg}
\caption{HEA formation based on $\Omega$ and the atomic size effect $\delta$}
\end{subfigure}
\begin{subfigure}{.85\textwidth}
\includegraphics[width=\textwidth]{theory/heaformation2.jpeg}
\caption{HEA formation based on $\Omega$ and the number of consituents $N$}
\end{subfigure}
\caption{Formation of HEAs based on $\Omega = \frac{T_\text{m} \delta S_\text{mix}}{|\Delta H_\text{mix}|}.$, the atomic size effect $\delta$, and number of constituents $N$. Figures adopted from \cite{hea2016_ch2}}
\end{figure} 
An important quantity in terms of characterizing high-entropy alloys, is the total number of valence electrons VEC. Derived from the work of Guo et al. on the phase stability of the \ch{Al_xCrCuFeNi_2} HEA, the VEC can be directly related to the crystal structure of high-entropy alloys. A lower VEC stabilize the BCC phase, while higher values stabilize FCC, in between is a mixture of the two. Specifically values greater than 8.0 stabilize FCC, and values bellow 6.87 favor BCC. However, these boundaries are not rigid when including elements outside of transition metals, exceptions has also been found for high-entropy alloys consisting of manganese. Although a heavy majority of reported high-entropy alloys that form solid solutions has been found to adopt simple cubic structures, such as FCC and BCC. In addition to the high-entropy silicides mentioned in the introduction, recent studies has reported HEAs in less symmetric structures, such as \ch{CoFeMnTi_x V_y Zr_z}, \ch{CrFeNiTiVZr} and \ch{CoFeNiTi} in HCP, as well as the \ch{Ti_{35}Zr_{27.5}Hf_{27.5}Ta_5Nb_5} HEA in the orthorhombic crystal system.  

\section{Core effects and properties}
In this section, we will summarize the discussion above into four core effects that can be used to explain high entropy alloys, and discuss some of the properties observed in various HEAs. The first core effect is called the "high-entropy effect", as explained in the previous section the high configurational entropy of HEAs compared to traditional solids or even binary alloys is central to stabilize the disordered phase ahead of intermetalic or strongly ordered structures. This effect can result in enhanced strength and ductility. From considerations of Gibbs free energy, we see that this effect is most prominent at elevated temperatures.

The second effect is the "severe lattice distortion effect" that arises from the fact that every element in a high-entropy structure is surrounded by non-homogeneous elements, thus leading to severe lattice strain and stress. The overall lattice distortion is additionally attributed to the differences in atomic size, bonding energies and crystal structure tendencies between the components. Therefore, the total lattice distortion observed in HEAs are significantly greater than that of conventional alloys. This effect mostly affects the strength and conductivity of the material, such that a higher degree of distortion yields greater strength and greatly reduces the electronic and thermal conductivity due to increased electron and phonon scattering. An upside to this is that the scattering and following properties become less temperature dependent given that it originates from the lattice rather than thermal vibrations. The concept of lattice strain in high-entropy alloys can be visualized as in figure 2.2.

\begin{figure} 
\centering
\includegraphics[width=\textwidth]{theory/latticeDistortionHEA.jpeg}
\caption{A schematic illustration of lattice distortion in high-entropy alloys, compared to conventional materials. Figure from \cite{owen_jones_2018}.}
\end{figure}
 
The two remaining effects, "sluggish diffusion" and "cocktail effect" can be summarized swiftly. The first is a direct consequence of the multi-component layout of high-entropy alloys that results in slowed diffusion and phase transformation because of the number of different elements that is involved in the process. The most notable product from this effect is an increased creep resistance. Lastly, we have the cocktail effect which is identical to the smoothie analogy mentioned previously, in that the resultant characteristics of a high-entropy alloy is a combination of both the individual elements and their interplay. This is possible the most promising concept behind high-entropy alloys, which fuels researchers with ambitions to discover highly optimized materials by meticulously combining and predicting properties from different elements. Examples of this can be the refractory HEAs developed by the "Air Force Research Laboratory" that significantly exceeded the melting points and strength of previous Ni or Co-based superalloys, by alloying specifically refractory elements such as Mo. Nb and W. Another example is the research conducted by Zhang et al. on the high-entropy system \ch{FeCoNi(AlSi_x)}, with ($0 < x 0.8$). In this HEA it was found that increased amount of either Al or Si lowered the saturation magnetization of the alloy. By tuning the relative amounts, it was found excellent properties for an $x=0.2$ HEA in terms of the magnetization, electrical resistivity and yield strength to produce a promising soft-magnet. The same was also found in \ch{Al_{0-2}CoCrFeNi} HEAS, where the addition of Al reduced the ferromagnetism of the alloy, and in \ch{CoCrCuFeNiTi_x} alloys where $x = 0$ was paramagnetic and $x > 0.8$ showed superparamagnetic properties. In general we find that that the saturation magnetism is mostly dependent on the contents and distribution of ferromagnetic elements such as Fe, Co and Ni while the addition of anti-ferromagnets like Cr could be difficult to predict. For example in the ferromagnetic HEA CoFeMnNiX, X = Al, Cr, Ga, Sn, studied in \cite{ZUO201710}, Mn pushed the material to the ferromagnetic phase, meanwhile addition of Cr pushed the material to a paramagnetic phase. Likewise in the equimolar system of \ch{CrMnFeCoNi} \cite{PhysRevB.96.014437}, the local magnetic moment of Cr was found to align antiferromagnetic, and the ferromagnetic character was attributed to local magnetic moments around Fe and Mn. 

As we have seen from the above examples, what makes high-entropy alloys a particularly interesting and promising field, is that they possesses the ability to be tuned for specific applications and properties, by testing specific combinations and distributions of different elements. In many ways, not indifferent to a smoothie or a cocktail.
\chapter{Density-Functional Theory}
\label{sec:label3}


\section{dft}


\chapter{Density-Functional Theory}
\label{sec:DFT}


\section{Brief review of Quantum mechanics}
\textbf{Begin by single particle function and derive the time independent equation or simply begin there. Then move on the many particle equation and introduce the Bloch theorems and such} 

\section{Fundamentals of DFT}


\section{Short-comings of DFT}


\section{Extensions of DFT}

\subsection{GGA}
\subsection{SCAN}
\subsection{Hybrid Functionals}




\part{Methodology and Implementation}
\chapter{Practical application of DFT}
\label{sec:Practical DFT}

In this section we will present the practical application and implementation of density functional theory in the study of materials science.

\section{The Exchange-Correlation functional}

From the former section, we know that the only peace of information we require to perform DFT plane-wave calculations are that of the exchange-correlation energy $E_{\text{xc}}[n]$. Given that the functional properties of a material is strongly related to the electronic density, the precision of DFT is very sensitive to the type of approximation to the density. The initial challenge is to balance the factors affecting the accuracy of the model to the computational cost.

The exchange-correlation functional is only exactly known for a homogeneous electron gas, however this is of limited use in real materials with variation in the electron concentration. An improvement on this is the local density approximation (LDA), in which the electronic density is set at each position according to the homogeneous electron gas at that position, in other words
\begin{equation}
    V_{xc}(\vec{r}) = V_{xc}^{\text{e gas}}[n(\vec{r})]
\end{equation}
This is seen as a successful approximation for the exchange-correlation energy in bulk materials, given that typically the electron density does not vary tremendously. However this method is not without flaws, most notably is the presence/degree of self-interaction from the fact that this term does not completely cancel out by only including the local environment. This may lead to artificial contributions to $V_{xc}$ and an overall inaccurately high electron density. The success behind this method most reliably lies in the low computational cost. 

The generalized gradient approximation (GGA) extends on the concept of LDA by also including the gradient of the electron density. 
\begin{equation}
    V_{xc}^{GGA}(\vec{r}) = V_{xc}[n(\vec{r}), \nabla n(\vec{r})].
\end{equation}
GGA is good for materials with a slowly varying density, but fails with large gradients. The GGA functional is implemented in two different methods, Perdew-Wang 91 (PW91) and Perdew-Burke-Ernzerhof (PBE). This method can further be extended to include the second order gradient, methods with this approach are called meta-GGA. Popular implementations include \textit{Modified Becke and Johnson} (MBJ) \cite{mbj} and \textit{Strongly Constrained Appropriately Normed} (SCAN) \cite{scan}. \textbf{Write a brief discussion of these, see for example \cite{scan2}}.

Both methods underestimate the band gap and wrongly predict the charge localization in cases of narrow bands or lattice distortion. A large part of this comes from the self-interaction of the Hartee potential. By combining the correlations and exchange of LDA/GGA with the exact exchange of the Hartree-Fock approximation, we can obtain much more precise calculations. This method was proposed by Becke as hybrid functionals, since the functional is a hybrid between the two. This method is superior in describing localized states, but comes at a significant larger computational cost. In order to reduce this cost, Heyd al et. split the Hartree-Fock exchange into short-range and long-range parts, in which calculations can adapt exact Hartree-Fock exchange for short-range (SL) and non-exact for long-range (LR). By introducing the parameter $\omega$ to adjust the order parameter of the method, we can express this method, called HSE (Heyd-Scuseria-Ernzerhof) \cite{hse06} as
\begin{equation}
    E_{xc}^{HSE} = \alpha E_{x}^{HF,SR}(\omega) + (1-\alpha)E_{x}^{PBE, SR}(\omega) + E_x^{PBE,LR}(\omega) + E_{c}^{PBE}
\end{equation}
\begin{itemize}
    \item Explain what the different notations mean, and the success + limitations of HSE compared to LDA and GGA.
    \item Write a paragraph on meta-GGA, such as SCAN and perhaps MBJ.
\end{itemize}
\textbf{Write more on HSE06, see references \cite{hf_bandgap}, \cite{hf_comparision}}

\section{Fundamental aspects of practical DFT calculations}
With the exchange-correlation functionals presented above, we now have everything in order to perform DFT calculations. To begin solving eq .., we need the single-electron wave-function, for a free electron this is a plane wave $\psi_k = Ae^{i\boldsymbol{k}\boldsymbol{r}}$. In a solid however, there exist a nonzero periodic potential $V(\boldsymbol{r}) = V(\boldsymbol{r} + \boldsymbol{R})$, the solution to the Shr\"{o}dinger equation is given by Bloch's theorem wich states that the solution takes the form
\begin{equation}
\psi_{\boldsymbol{k}}(\boldsymbol{r}) = u_{\boldsymbol{k}}(\boldsymbol{r})e^{i\boldsymbol{k}\boldsymbol{r}},    
\end{equation}
where $u_{\boldsymbol{k}}(\boldsymbol{r}$ is a bloch wave with identical periodicity to the supercell. And $\boldsymbol{k}$ is the wavevector. Along with eq(above), problems in DFT are solved in k-space or reciprocal space for convience sake. For instance a great deal of DFT calculations revolve around solving the integral 
\begin{equation}
    g = \frac{V_{\text{cell}}}{(2\pi)^3} \int_{\text{BZ}} g(\boldsymbol{k})d\boldsymbol{k},
\end{equation}
with BZ denoting that the integral be evaluated for all $\boldsymbol{k}$ in the Brillouin zone. This integral can be approximated by evaluating the integral at a set of discrete points and summing over the points with appropriatly assigined weigts. A larger set of points leads to more exact approximations. This method is called Legendre quadrature. The method for selceting these points in reciprocal space was devolped by Monkhorst and Pack in 1976, and simply put requieres a amount of kpoints in each direction in reciprocal space, in the form $N x N x N$. Recalling that reciprocal space is inverse to regular space, supercells with equal and large dimensions converge at smaller values of N, and inversly for cells of small dimsion. In supercells with different length axis, such as hexagonal cells, we use the notation $N x N x M$, where $M$ relate to the distincntly different axis. The amount of kpoints required can be fruther reduced by utulizing the symmetry of the cell, in which we can exactly approximate the entire BZ by extending a lesser zone through symmertry. This reduced zone is appropriartly named the irreducible Brillouin zone (IBZ). 

Metals in particular requiere a large set of kpoints to acchive accurate results. This is becouse we encounter discontinuies functions in the Brillouin zone around the fermi sufrace where the states discontinusly change from occupied to non-occupied. To reduce the cost of this operatin, there are two primary methods, tetrhaedon and smearing. The idea behind the tetrahedon method is to use the discrete set of k-points to fill the reciprocal space with tethraeda and interpolate the function within each tethraeda such that the function can be integrated in the entire space rather than at discrete points. The latter approach for solving discontinuos integrals is to smear out the discontinuity and thus transforming the integral to a continous one. A good analogy to this method is the fermi-dirac function, in which a small variable $\sigma$ transform a step-functino into a continious function that can be integrated by standard methods.

In addition to the number of kpoints, there is one more distinct parameter that must be specified in DFT calculations, namely the energy cutoff, or $E_{\text{cut}}$. This parameters arise from the Bloch function described previosly. In which $u_{\boldsymbol{k}}(\boldsymbol{r})$ was a bloch wave with the same periodicity as the supercell. This implies that the wave can be expanded by a set of special plane waves as
\begin{equation}
    u_{\boldsymbol{k}}(\boldsymbol{r}) = \sum_{\boldsymbol{G}} c_{\boldsymbol{G}}e^{i\boldsymbol{G}\boldsymbol{r}},
\end{equation}
where $\boldsymbol{G}$ is the reciprocal lattice vector. Combining this with eq ..(first eq for blcoh function) we get 
\begin{equation}
    \psi_{\boldsymbol{k}}(\boldsymbol{r}) = \sum_{\boldsymbol{G}} c_{\boldsymbol{k} + \boldsymbol{G}}e^{i(\boldsymbol{k} + \boldsymbol{G})\boldsymbol{r}}
\end{equation}
The consequense from this expression is that evaluating the wavefunction of an electron at a single $k$ point demand a summation over the entirity of reciprocal space. In order to reduce this computational burden, we can introduce a maximum paramater $E_{\text{cut}}$ to cap the calculations. This is possible becouse eq ..(above) is the solution of the Shr\"{o}dinger equation with kinetic energy 
\begin{equation}
    E = \frac{\hbar^2}{2m}|\boldsymbol{k} + \boldsymbol{G}|^2.
\end{equation}
Seeing as the solution with lower energies are the most interesting, we can limit the calculations of eq ..(2 above) to solutions with energy less than $E_{\text{cut}}$ given bellow
\begin{equation}
    E_{\text{cut}} = \frac{\hbar^2}{2m}G_{\text{cut}}.
\end{equation}
Thus, we can reduce the infinitly large sum above to a much more feasable calculation in 
\begin{equation}
    \psi_{\boldsymbol{k}}(\boldsymbol{r}) = \sum_{|\boldsymbol{k} + \boldsymbol{G}| < G_{\text{cut}}} c_{\boldsymbol{k} + \boldsymbol{G}}e^{i(\boldsymbol{k} + \boldsymbol{G})\boldsymbol{r}}
\end{equation}

\textbf{A summary on kpoints and ENCUT, plus a discussion on nummerical convergence and how to select kpoints and ENCUT}


A final consideration to how DFT is applied in practise is how the core electrons are handled. Tightly bound core electrons as opposed to valene electrons demand a greater number of plane-waves to converge. The most efficient method of reducing the expenses of core-electrons are so-called pseudopotentials. This method works by approximating the electron density of the core elecrons by a constant density that mimic the properties of true ion core and core electrons. This density is then remained constant for all subsequent calculations, ie only considering the valence electrons while regarding the core electrons as frozen-in. There are currently two popular types of psudopotentials used in DFT, so-called ultrasoft psudopotentials (USPPs) devoplped by Vanderbilt, and the projecter augmented-wave (PAW) method by Bloch \cite{PAW1}, \cite{PAW2}. This project will exclusively apply the latter. 

\section{Self-consistent calculation}

\begin{figure}
\centering
\includegraphics[scale=.3]{theory/selfConsistentDFT.jpeg}
\caption{Self consistent iteration of a DFT calculation. Figure adopted from ..}
\label{sfDFT}
\end{figure}
\chapter{Computational details}
\label{sec:Computation}

This section is intended to provide the necessary details for reproduction of results to be presented later on. First we begin by describing the software used for the project. 

\section{Vienna Ab initio Simulation Package}
This software, often referred to as VASP is a package for ab initio quantum mechanics calculations using the projected augmented wave method and plane wave basis set. The intended methodology is DFT, but have been extended for methods post the original DFT-formulation. Calculations with VASP was carried out on the supercomputer fram, with allocated time and resources provided by Uninett Sigma2,\textbf{add reference!}.

The structure of VASP rely on a set of input files and output files from the calculation, the input files required to perform a DFT computation in VASP are the following:
\begin{itemize}
    \item INCAR - this file provide the tags responsible for different methods, algorithms, parameters etc.
    \item POSCAR - this file is related to the crystal structure of the system
    \item POTCAR - What psudopotential that is used
    \item KPOINTS - A file containing information on what KPOINTS will be used
    \item jobfile - This file contains information for the supercomputer regarding resources and such.
\end{itemize}
The capitalization displayed above is directly related to the requirements of the file system in the VASP/fram collaboration. Some important output files are:
\begin{itemize}
    \item CONTCAR - The relaxed crystal structure after finalized calculation
    \item CHGCAR - This file contains the electron density after calculation
    \item EIGENVAL - Contains the solutions to the Kohn-Sham eigenfunctions
    \item DOSCAR - Information on the Density of States
    \item OUTCAR - Contains a list of all other information.
\end{itemize}
In this project, we began the calculation of every individual structure by testing the convergence of total energy with respect to the number of k-points and cutoff energy. In VASP, the latter can be specified by setting the tag "ENCUT" in the INCAR file, we found 300 eV to yield productive results in terms of convergence and computation time for total energy calculations, and 400 for ionic+volume relaxations. Regarding the number of points in the reciprocal space, we carried out a great deal of simulations on numerous structures with distinct crystal structures and corresponding supercells, for this reason we employed a number of different sets of k-points depending on the structure. Typically the number of points ranged from a 2x2x2 mesh to 4x4x4 mesh. With the smaller being required for hybrid functionals to converge. 

Upon realizing the convergence parameters, the structures were allowed to relax both the ionic positions, and cell volume with the quasi-newton method and a convergence criterion of $1E-2$ for the forces and $1E-5$ for the total energy. However, the symmetry of the structure was forced constant by the use of vasp-std-noshear. This process was repeated two times before performing one final total energy calculation with GGA functional, and one with the SCAN functional, increasing the convergence criterion to $1E-6$. In certain cases, we also attempted to perform calculations with the hybrid functional HSE06.

The specific tags, algorithms, parameters and options of VASP that was in use throughout this project can be found at our GitHub address, but in particular we would like to cover specifically two factors/tags. Firstly is the tag ISPIN that correspond to the magnetism of the system. Considering this study revolved heavily around magnetic elements such as iron and Nickel, we used ISPIN=2 which allow for co-linear spin-polarized calculations. However, there are many more magnetic orientations the system can adopt besides co-linear, therefore the final total energies we found may not be the true lowest energies. But given the allocated duration and resources of this project, this is a understood consequence. Secondly is the type of smearing that was used for the different calculations. The preferred method for accurate total energies and density of states in semiconductors is the tetrahedron method, and for accurate forces in metals the Methfessel-Paxton method is recommended. However, our system contains both metals and a large portion of Si. For this reason we used a combination of smearing methods. For the relaxation and minimization of forces, we used gaussian smearing with smearing width $\sigma = 0.05$, as this method provide accurate forces in both metallic and semiconducting materials. And to calculate the total energy and DOS, we used the tetrahedron method, as recommended. One interesting result of this project, is that we were not able to converge calculations with hybrid functionals using the tetrahedron method and was thus forced to adopt the Gaussian method in this case additionally. 

\textbf{Band structure/DOS and band-unfolding?}

\section{Generation of SQS}
The generation of special quasi-random structures as described in section .., was done by utilizing the Temperature Dependent Effective Potential (TDEP) method. This package, devolved by Olle Hellman, offers a wide range of tools primary intended for studies of finite temperature lattice dynamics. In this project we utilize the program generate-structure within the TDEP package to construct SQS's. The work of TDEP is the result of an unpublished PHD thesis by Nina Shulumba \textbf{(Insert citation)}, thus the documentation on the software and generate-structure script is limited, please refer to the original author for more information. 

In this project, we constructed SQS's by first transforming the cif-files of a given initial structure, for instance that of $FeSi_2$, to a primitive unit cell. The SQS's was generated by the same principles explained in section .., for each structure we created 5 distinct SQS's of an equal size under the constraint that the 3d atoms be distributed eqvimolar in the system. Precise file formats and such can be found at GitHub. Another approach could have been to construct SQS's of specific cell counts instead of total number of atoms, however this quickly lead to extremely large supercells, up to 256 atoms, that simply would not converge to our best efforts. 

We began by studying high-entropy silicides by alloying 3d-metal silicides such as $Fe_2Si$ by Cr, Fe, Co, and Ni to construct a $(CoCrFeNi)_2Si$ alloys. From this point we varied the distribution and type of elements in an attempt to locate high-entropy silicides with semiconducting properties, but remained within \textit{quaternary} 3d silicides. 

\textbf{Insert figure of SQS's before relaxation, for some of the structures and give a brief summary.}

\section{Band-structure?}

\section{Utility scripts}
During the course of the projects lifetime, several shell and python scripts was developed by myself and/or provided to me by my supervisor Ole Martin Løvvik and his team of researchers at Sintef. These can be located at the GitHub address :...
 
\part{Results and Discussion}
\textbf{Trenger jeg denne biten? Evt hvor?}
In this one year long project, we have collected results of a great number of materials with various structures and compositions. The initial experimentation was based on high-entropy silicides of the $Fe_2Si$ unit cell, created from the special quasi-random structure approach as described above. Despite the non-semiconducting character of this compound, we worked under the idea that the extraordinary properties that have been observed in high-entropy alloys through effects such as the cocktail effect, we could discover specific combinations of elements that would yield a semiconductor. In addition, the ratio between silicon atoms to metals allowed us to create nearly eqvimolar high-entropy alloys. 

Following this attempt, we transistioned into studying high-entropy silicides based on well known semiconducting 3d silicides such as $\beta-$\ch{FeSi2}, \ch{CrSi2} and \ch{MnSi_{1.75}}. The main outcome of this project is that for all 4 different starting silicides, we could only produce high-entropy silicides from one unit cell, furthermore in this cell only one specific compositions of elements was semiconducting. This was \ch{Cr_{0.25}Fe{0.25}Mn_{0.25}Ni_{0.25})Si2}, here-in CFMN, in the $\beta-$ \ch{FeSi2} crystal structure.  

This section will be structured in the following manner, firstly we will investigate the CFMN (fesi2) compound and various permutations of the composition. Thereafter we will look at other possible compositions of fesi2 based high-entropy silicides, and lastly test the CFMN composition in other crystal symmetries. In final we will present an overview of the complete study and the various compounds that have been investigated in order to propose promising directions and guideline future research directions in this field. In this way, we aim to understand the uniqe properties of CFMN (fesi2) and why this particular compound is semiconducting compared to the other testes structures in this project. Properties we will cover is the overall stability by total energy and corresponding enthalpy of formation, the magnetic properties and which elements contribute to the magnetism. But in majority, we will look at the band gap and related properties, as this is the main motivation and distinction of the study.    

\chapter{The good (CFMN fesi2)}
\label{sec:good}

\textbf{Add figure of LDOS around Ef for SQS B to compare with sqs D. Also change the name from local DOS to projected DOS and explain the assumption regarding 3d electrons and why we did not include plots of the local DOS.}
$\beta-FeSi_2$ in the orthorombic cmce crystal lattice is a well known semiconductor with an experimentally measured band gap of around 0.8 ev \textbf{cite}, the nature of the band gap is under debate, all though most ab inito studies point to an indirect gap, experimental work indicate a direct gap. From our own DFT calculations, we find an indirect band gap close to 0.65 eV with PBE. This is in good agreement with other meassurements from ab intitio studies \textbf{cite materials projects, other studies.} 

The density of states and charge density of bulk $\beta-$\ch{FeSi2} from PBE calculations can be seen in figure .., ..  From the figures we observe a clear band gap and semiconducting character. Moreover, we note from the density of states that the gap is identical in both spin channels, indicating that this material is diamagnetic. We find this to be true from the written magnetization in VASP, this also is in agreement with relevant literature \textbf{cite}. \textbf{Find reference for stability and $\Delta H^0$.} Finally, the enthalpy of formation of this compound is -18.6583 eV.

\section{CFMN Eqvimolar distribution}

\subsection{Introduction}

The CFMN alloys of the fesi2 unit cell alloys can be seen in figure ... The supercells consist of 48 atoms, 16 of which is evenly distributed between Cr, Fe, Mn, and Ni, the remaining 32 sites occupied by silicon atoms. Bellow in table .. we list the total energy per atom (Toten), final magnetic moment per atom (Mag), and the band gap of the five distinct SQSs corresponding to the CFMN (fesi2) compound. In addition we include the mean and standard deviation of the values, plus the enthalpy of formation. For simplicity, we denote the SQSs as A, B, C, D and E.

\begin{table}[H]
\centering
\begin{tabular}{@{}cccc@{}}
\toprule
Structure  & Toten (eV) & Mag (?) & Band gap (eV) \\ \midrule
\textbf{A} & −6,6080                & 0.0833                    & 0.0280        \\
\textbf{B} & −6,6138                & 0.0833                    & 0.0523        \\
\textbf{C} & −6,6063                & 0.0834                    & 0.0344        \\
\textbf{D} & −6,6155                & 0.0833                    & 0             \\
\textbf{E} & −6,6089                & 0.0833                    & 0.0495        \\ \midrule
\textbf{Mean} & -6.6105 & 4.0000 & 0.0328    \\
\textbf{Std} & 0.0039 &  0.0000 &  0.0210 \\
\textbf{$\Delta H_{mean}^0$} & -11.5000 eV & - & - \\ \bottomrule
\end{tabular}
\caption{Total energy per atom, final magnetic moment, band gap (GGA) and formation enthalpy of $Cr_4Fe_4Mn_4Ni_4Si_{32}$ SQSs based on $FeSi_2$}
\label{table:fesi2_summary}
\end{table}  

\textbf{Write a section on magnetism in method}
From a first glance, we observe very similar properties between the SQSs regarding both the total energy and final magnetic moment. Comparing to bulk \ch{FeSi_2}, this compound is both less stable, from the enthalpy of formation, and magnetic. For the magnetic character of the compound, we performed self-consistent total energy calculations with three diffrent magnetic configurations, non-magnetic (ispin=1), colinear magnetism with the initial magnetic moment equal to 1 times the number of ions, and lastly two times N ions. Of the three starting positions, we found the two latter to yield near identical total energies, with the middle seting winning out in some SQSs. The consistent magnetic moment between the 5 supercells is excpected seeing as all 5 structures consist of equivalent elements. The magnetic moment observed is solely attributed from 3d electrons and in particular those of chromium and manganese atoms. 
 
\subsection{The band gap} 
 
The most interesting property of these SQSs is in fact the band gap. We note a mean band gap of about 0.03 eV, much lower than 0.65 eV of bulk $FeSi_2$. But a band gap in this smaller range makes for  excellent application in for instance thermoelectrics. The gap is seen in 4 out of 5 SQSs, but surprisingly not in the most stable arrangement (D), the largest gap observed is about 0.05 eV from structure B, which is slighlty bellow D in terms of total energy, but still a way above the mean energy. Similar to the bulk material, also these band gaps are indirect, the transitions are listed bellow in table .. .     

\begin{table}[H]
\centering
\begin{tabular}{@{}ccc@{}}
\toprule
Structure  & Gap (D/I) & Transition                              \\ \midrule
\textbf{A} & I         & (0.500,0.333,0.500) $\rightarrow$ (0.500,0.000,0.000)  \\
\textbf{B} & I         & (0.250,0.000,0.250) $\rightarrow$ (0.000,0.000,0.000)  \\
\textbf{C} & -         & (0.500,0.000,0.500) $\rightarrow$ (-0.250,0.333,0.500) \\
\textbf{D} & I         & -                                        \\
\textbf{E} & I         & (0.000,0.000,0.000) $\rightarrow$ (0.250,0.000,0.250)  \\ \bottomrule
\end{tabular}
\caption{Band gap transition of CFMN (fesi2) SQSs with PBE functional}
\end{table}

A very useful method to extract information regarding the band gap of a material is to plot and study the band structure, however this is not as insightful when considering large supercells consisting of several elements and a  large number of energy bands. The solution to this is normally to do a band unfolding, but given the complex structure and implementation of these SQS is VASP we where not able to do either. Instead we can study the band gap by firstly observing the density of states, in figure .. we plot both the total density of states (TDOS) and the local density of states (LDOS) of SQS D.

\begin{figure}[H]
	\centering
	\includegraphics[width=\textwidth]{results/fesi2/D_TDOS.png}
	\caption{Density of states SQS D CFMN (fesi2) from PBE calculation}
\end{figure}

\textbf{Rewrite later}
From TDOS we learn that D is in fact a half-metal with a sizable band gap in the spin $\uparrow$ channel, and a display a metallic character in spin $\downarrow$. Considering now the projected density of state plotted in figure 8.2 we observe that the lower energy states are filled by silicon atoms. At slightly higher energies there is evidence of Silicon and TM hybridization. In both spin channels Ni lie at the lowest energies of the 3d elements, followed by iron and then manganese and chromium very close to the fermi energi. Above the fermi enery there is more of an equal contribution from all elements, however slightly above Ef particularly iron and chromuim show a distinction in spin up and down respectfully.  At higher energies, the PDOS is attributed to Si and Cr in both spin states, while elements such as Ni, Fe and Mn have a lesser role. In this figure we assume that the TM's correspond to 3d electrons, and that in the case of silicon the s electrons are most evident for the lower energies, while the hybridzation comes from p-electrons in Si and d-electrons in TMs. Because of the large number of elements in this structure we find the local density to be tedious and difficult to interpret. However, several recent studies on TM silicides point to this sort of local density of states \textbf{Cite references.} 

\begin{figure}[H]
	\includegraphics[width=\textwidth]{results/fesi2/D_PDOS.png}
	\caption{Projected density of states SQS D CFMN (fesi2) from PBE calculation}
\end{figure}

\begin{figure}[H]
	\begin{subfigure}{\textwidth}
		\includegraphics[width=\textwidth]{results/fesi2/B_TDOS.png}
		\caption{Density of states SQS B CFMN (fesi2) from PBE calculation}
	\end{subfigure}
	\begin{subfigure}{\textwidth}
		\includegraphics[width=\textwidth]{results/fesi2/B_PDOS_Ef.png}
		\caption{Projected density of states SQS B CFMN (fesi2) from PBE calculation}
	\end{subfigure}
\end{figure}

Above we have plotted the density of states and projected density of states around Ef of SQS B, both clearly showing a band gap in both spin channels. Similarly, also this SQS have a spin-polarized band gap, in spin up we see a band gap of around 0.3 eV, while the spin down states have a lesser band gap of 0.05 eV. This is a common trend of all SQSs excluding D of this composition. In table .. we list the band gap in both spin channels and the resulting total gap of all 5 SQSs. From the PDOS we find that one key distinction to SQS D is the part of manganese at energies right above Ef in spin down, as opposed to the semiconducting structures.

\begin{table}[H]
\centering
\begin{tabular}{@{}cccc@{}}
\toprule
Structure  & Spin-up & Spin-down & Total  \\ \midrule
\textbf{A} & 0.0814  & 0.0522    & 0.0281 \\
\textbf{B} & 0.2932  & 0.0523    & 0.0523 \\
\textbf{C} & 0.2355  & 0.0343    & 0.0343 \\
\textbf{D} & 0.3386  & 0         & 0      \\
\textbf{E} & 0.3078  & 0.0495    & 0.0495 \\ \bottomrule
\end{tabular}
\caption{Band gap (eV) with PBE in spin up and spin down channels of CFMN (fesi2) SQSs}
\end{table}

The density of states is a widely used and insightful tool to visualize the band gap of a solid. Additionally we can also obtain information on the magnetic character by the spin polarization of the band gap. In the context of DFT and VASP however, the DOS include several factors that may contribute to inaccurate and sensitive results. As mentioned in section .., the type of numerical smearing is paramount for accurate DOS calculations. In this project we experienced large differences between calculations from gaussian and TBS smearing in relation to the band gap and DOS, this will be covered in more detail later. Moreover the DOS is very sensitive to computational factors such as the number of points in the DOS (NEDOS in VASP) and the number of k-points (to solve the DOS integral, see section ..). For example, the band gap in structure C could only be seen in the density of states when increasing the number of points in the DOS from 2401 to 20000 points. This is shown bellow in figure blabla, where we plot the density of states around the fermi energy, denoted by the strippled red and blue lines, relative to the density of states with 2401 points and 20000 points respectfully, all other parameters remained unchanged, it should however be noted that the second calculations applied the charge density calculated by the former for quicker convergence. 

\begin{figure}[H]
	\includegraphics[width=\textwidth]{results/fesi2/C_DOS.png}
	\caption{Density of states of SQS C with 2501 points vs 20000 points in the density of states.}
\end{figure} 

Despite of the higher accuracy of the greater number of points, we continue to perform calculations with 2401 points in most calculations, mostly down to the increased workload for analyzing and producing DOS related results with such a large number of points, and the use of vaspkits tools. 
 
A more secure method of evaluating the band gap is to consider the Kohn-Sham eigenvalues. The eigenvalues are provided for all energy bands for the given number of k-points used in the calculation, with listed energies and corresponding occupancy in both spin channels. The values listed in table (..) above was calculated from the eigenvalues. In addition to validate and provide an additional measure to the density of states, we can qualitatively differentiate SQS D. For certain k-points the occupancy does not transition from 1 to 0 directly between two bands, but rather contain one or more partially occupied bands in between (\textbf{Visualize? type fermi-dirc plot}), however only in the spin down channel. If we were to neglect these partially occupied states and only consider bands where the occupancy is above 0.99 or bellow 0.01, the band gap of structure D remain consistent in spin up, but we now observe a band gap of around 0.05 eV in the spin down channel resulting in a total band gap in the structure. Again, this would have been extremely insightful to investigate with the help of a band structure diagram.

\subsection{Meta-GGA and hybrid functional}

As expressed previously, in this work we involve 3 level of depths GGA (PBE), meta-GGA (SCAN) and hybrid functionals (HSE06) to determine the band gap of the SQSs. In table .. bellow we list the respective band gaps of these methods for all 5 SQSs of CFMN (fesi2). Note that all calculations is done with TBC smearing.

\begin{table}[H]
\centering
\begin{tabular}{@{}cccc@{}}
\toprule
Structure  & PBE    & SCAN   & HSE06  \\ \midrule
\textbf{A} & 0.0281 & 0.0000 & 0.0207 \\
\textbf{B} & 0.0523 & 0.0890 & 0.1808 \\
\textbf{C} & 0.0344 & 0.0690 & 0.0196 \\
\textbf{D} & 0.0000 & 0.0000 & 0.0000 \\
\textbf{E} & 0.0495 & 0.1048 & 0.0133 \\ \bottomrule
\end{tabular}
\caption{Band gap of CFMN (\ch{FeSi2}) SQSs with GGA (PBE), meta-GGA (SCAN) and hybrid-functionals (HSE06).}
\end{table}

\textbf{Need a comment on why we use the SCAN functional exactly, answer: Considered accurate, fast, and allows for testing on a level between gga and hybrid. Write this in the method section.}

The most obvious result of table .. is that aside from SQS A, all 3 methods agree on the presence of the band gap. This in itself is a very positive result for this project, as the primary motivation is based simply on locating semiconducting high-entropy silicides and thus the agreement of 3 different methods on the same structures is most welcome. On the other hand, it's clear that the actual size of the gap is under some debate. We note the largest observed band gaps is largely associated with the SCAN functional, compared to PBE calculations this result is very in line with what is expected by involving more complex factors in the calculations, as discussed in section .. In contrast, by the same argument we would not expect that par SQS B, the overall smallest band gaps is found with the well-proven hybrid functional HSE06, as shown in table .. The results associated with the HSE06 functional will be covered in more detail in the subsequent section, for now lets consider SCAN. 
 
\paragraph{SCAN \\}
For the most part, the results with SCAN meta-GGA prove similar to PBE, as was the case in the bulk material. The one exception is in SQS A. In this case the SCAN calculations result in a metallic character as opposed to the 0.03 eV band gap from PBE. Upon investigating we discover that the eigenstates is riddled with both partial occupancy and so-called non-physical values. If we were two neglect these, we find a band gap of 0.031 eV, very in line with the PBE result \textbf{Should I include this? I don't know why or what this means. And I don't really wanna spend a significant portion on one result with SCAN.} 

Other noteworthy concerns about the SCAN functional is apparent in the results of SQS C and D. 
\textbf{Plot DOS C with PBE and SCAN side by side or like NEDOS around Ef to show the difference. Also make one for E to show how SCAN decrease the spin up gap and increase the spin down gap.}

\paragraph{HSE06 \\}
\textbf{Wait for jobs to finish: lesssmear and ismear0}
As stated above, the measured band gaps with the HSE06 functional was less than that of PBE and SCAN for most of the tested SQSs. Hybrid functionals as described in section .. is computationally demanding, but comes with superior accuracy for band gap measurements, and the HSE06 functional in particular is on the top of the list. For this reason, one would in general expect larger band gaps compared to GGA or meta-GGA calculations, as highlighted in .. \textbf{cite?} The one exception we observed to this trend is in SQS B, here the band gap increase from 0.05 eV to 0.09 eV and 0.18 eV from PBE to SCAN to HSE06. One possible reason behind the abnormally large gap can originate from the small number of k-points we had to employ in order for the calculations to converge. Recalling that the gap transition in in the PBE calculation was (0.250,0.000,0.250)-(0.000,0.000,0.000), compared to the hybrid functional we now see that the transition is between k-points (0.500,0.000,0.000) and (0.000,0.000,0.000). Moreover, the point (0.250, 0.000, 0.250) in k-space is not included in the hybrid functional due to the narrow mesh (this we read from the IBZKIT file in VASP). Thus it's a possibility that the large gap is caused by the fact that the minimal gap is not encapsulated by the k-points in the HSE06 calculation. However we also see this trend in the other SQSs, but despite of the different transistion in k-space, these structures find lesser band gaps with the HSE06 functional compared to PBE. Additionally, we find similar results in the bulk $\beta-$\ch{FeSi2} structure. In this calculation we applied the same number of k-points for HSE06 as for PBE and SCAN. Nevertheless we find a much larger band gap of around 1.5 eV with HSE06, as opposed to 0.65 eV with both PBE and SCAN, and as mentioned before the two latter is in much better agreement with experimental results and ab intio work on the band gap of $\beta-$\ch{FeSi2} \textbf{cite materials project, other articles}. Additionally also in this case, the transition vary between functionals. PBE: (0.000,0.000,0.000)-(0.000,0.000,0.250), and HSE06: (0.000,0.000,0.000)-(0.000,0.000,0.500). \textbf{Include band-diagram for bulk fesi2?} 

Aside from SQS B, we find generally good agreement between HSE06 and PBE calculations. In A  we notice that the 0.02 eV band gap of tabel .. stems from a 0.7 eV gap in spin up and 0.02 eV in down. Likewise SQSs C, D, and E all exhibit large band gaps in spin up, 0.17, 0.37 and 0.55 eV respective, and corresponding very narrow gaps in spin down equal to 0.032, 0, and 0.013 eV. If we compare to the listed spin gaps from PBE in table .., we see that the band gaps of HSE06 typically compares or exceeds in spin up, and lessen in spin down, except in B (0.29 eV and 0.18 eV). 

\textbf{Include here a figure or number on the computation time between PBE, SCAN and HSE06.}
\textbf{Figure comparing the DOS between PBE, SCAN, HSE06 for one SQS?}

\textbf{Rewrite/reconsider this paragraph, is it needed? How can I write this more concise? Figure?}
One final point we would like to cover in the discussion of HSE06 calculations of this system, and generally in this project, is the effect of smearing on the reported band gaps. From the method section, we know that TBC smearing is the preferred choice for accurate density of states and total energy calculations of semiconductors, alike we know that this method is unfitting to calculate the forces in metals. As discussed in the methodology section, hybrid functionals proved difficult to converge for such composistionally complex structures, thus we were forced to initially calculate the charge density from the HSE06 functional with did a self-consistent calculation with gaussian smearing and smearing width of 0.05 eV. Thereafter reuse the calculated charge density for subsequent hybrid calculations with TBC smearing. Using SQS A as an example, from the first run (Gaussian), the band gap is 0.15 eV, (0.78 up and 0.15 down). However the eigenvalues contain defect states and the band gap is not observable from the density of states. Next we can reapply the charge density to perform an additional HSE06 calculation with gaussian smearing, but reducing the smearing width from 0.05 eV to 0.005 eV. Now we find a new gap of 0.1 eV (0.21 up and 0.1 down), with no defects in the eigenstates, and apperant in the density of states. In cases where we find conflicting results between the eigenvalues and density of states we rely on the script bandgap.py provided in the pymatgen package, refer to section .. for a description. With this we only report a band gap for the HSE06 calculation with TBC smearing, note that this method return the same value of the gap as well. As another example lets consider SQS B. In contrast, the nummerical smearing does not appear to impact the band gap of this structure. We find from HSE06 simulations with gaussian smearing of both 0.05 and 0.005 eV smearing width to yield results around 0.28 eV and 0.18 eV in spin up and down. But alike SQS A, the larger smearing width comes with a few defect states in the spin down channel and additionally can not be seen in the density of states. However, particular of this structure is that the bandgap.py script validate the calculated total band gap from the eigenvalues in all three calculations. Aside from this abnormalty, the other SQS similar to A find some similarities between smearings, but only TBC was validated with bandgap.py, furthermore the DOS does not with the same clarity reproduce the calculated band gap from the eigenvalues in calculations done with gaussian smearing compared to TBC. \textbf{Create figure/subfigure of the DOS of hybrid/smear/smear5 calculations to illustrate the above point, maybe A}

We see from the above examples that as most studies and articles state, that TBC smearing is superior in terms of accurate total energy and DOS calculations of semiconductors. Similar to how TBC produce inacurate forces of metals, in several cases in this project we relaxed the structures with gaussian to forces bellow 1E-2, but subsequent calculation with TBC in certain cases resulted in forces above 0.1, without making any geometric alteration to the previously relaxed cell \textbf{(Include examples?)}. On the grounds of these factors we can report good agreement between our own results and the theoretical advice regarding numerical smearing in DFT studies \textbf{Insert refrences}

We see that the band gap of the high-entropy silicide vary between both PBE, SCAN and HSE06. From the SCAN functional we found several cases that disagreeed with the PBE results, see SQS A, C and D. Combining this with the popularity and wide-spread application and reliability of the PBE functional. See for example materials project, that exclusivly list PBE band gaps, and other relevant studies \textbf{refrences}. We put the most faith in the PBE results. An additional point is that GGA is known to underestimate band gaps, due to the concepts described in section .., therefore if we find a gap with PBE, the real material would most probably also have a band gap, and a larger one at that.

Regarding the HSE06 functional, the inaccuracy shown for the bulk material is concering, escpecially considering the lack of experimental baselines in this study to compare and measure our results after. However, generally we find much better cohesion between PBE and HSE06 compared to SCAN for the 5 supercells, both methods predict semiconductors with heavy spin polarization in the spin up direction, par B. The fact that all 3 functionals and five structures for the most part agree on the presence of a semiconductor is a overwhelmingly positive result in itself, that allow us to state with high certainity that this compound is in fact a semiconductor, or we may label the compound as a half-metal or spin-gapless conductor from the registered spin dependence. A qualitative study on the exact band gap would demand a much greater scope as there are many factors affecting the value that we have neglected. One of these is the randomness involved with SQSs. For instance, by increasing the SQS size, ie number of atoms in the supercell, we found again different band gaps, but still, the presence and characther of the compound was consistent. \textbf{Include this? Table?} To draw any meaningfull conclusions on the size of the band gap would requiere us to both increase the number of SQS's of the composistion due to the obseved variation in the band gap between the 5 tested supercells, and as well for different supercell sizes to obtain some sort of convergence of the band gap. On the other hand, if we go by the most stable configuration, then this compound would be labeled as a half-metal from the results of SQS D.  

\subsection{Probability distribution functions and charge density}

In this final segment on the CFMN (fesi2) alloys we will include the probability distribution functions and charge density, which will be usefull for later comparisons. We only include the results of SQS B and D in this section, as we saw little variation across the five SQSs.   
 
\begin{figure}[H]
	\centering
	\includegraphics[width=\textwidth]{results/fesi2/D_PDF2.png}
\end{figure}

\begin{figure}[H]
	\centering
	\includegraphics[width=\textwidth]{results/fesi2/B_PDF.png}
\end{figure}

From these figures there is a lot of useful information to extract. With the aid of the ICSD (insert citation), we can compare the values of figure .. to the expected PDFs based on a number of experiments from a host of different compounds. As our compound contain a total of 15 different bonds, comparing each one to the ICSD values would be an exhaustive process. For our purpose we are satisfied by comparing the 4 different metal-Si bonds and note ourselves of key distinctions. We find that the preferred bond-length of TM-Si is observed at two values, the most dominant being the shorter of the two. For Fe-Si these are between 2.25-2.75 and 4-5, Mn-Si 2.25-2.75 and 3.5-5. Ni-Si lie between 2.25-2.5 and 3.85-5 and Cr-Si between 2.35-2.65 and 4-5.
Clearly, the PDFs of the alloys are in good agreement with the listed values for Tm-Si bonds, with the most occurring bond length falling at around 2.4 Å for all TMs, and lesser occurrence between 4.0 - 4.5 Å. The relative height of the peaks follow a similar trend, Fe-Si, Mn-Si, and Cr-Si all lie close to 8 for the first peak at 2.4 Å, and Ni-Si slightly bellow around 7. Moreover we note that the Fe-Si occurance at 2.4 Å is lower in SQS B compared to D \textbf{More on the PDFs?}

Bellow we show the calculated charge density (from PBE) of structure B (left) and D (rights).
\begin{figure}[H]
	\begin{subfigure}{.5\textwidth}
		\includegraphics[width=\textwidth]{results/fesi2/B_CHGCAR.jpg}
		\caption{Structure B}
	\end{subfigure}
	\hfill
	\begin{subfigure}{.5\textwidth}
		\includegraphics[width=\textwidth]{results/fesi2/D_CHGCAR.jpg}
		\caption{Structure D}
	\end{subfigure}
\end{figure}

\section{Permutations}

\textbf{Analyze LDOS, PDFs and CHGCAR of some SQSs}

Up until this point we have investigated the structure CFMN (\ch{FeSi2}). More specifically we have looked at the center of a quasiternary pahse diargram. In this section, we aim to exapand our search of this diargram by generating SQSs slightly away from eqvimolar distribution of 3d elements. In table (bellow) we list the mean total energy and magnetic moment per atom with standard deviation and the enthalpy of formation of 4 permutations of the CFMN (fesi2) compound. Ideally we would alter one element at a time, but the TDEP implementation insist in also reducing Nickel to stay consistent with the 48 atom supercells. Thus we reduce Ni and one additional element per permutation.

\begin{table}[h!]
\centering
\begin{tabular}{@{}cccccc@{}}
\toprule
       & \multicolumn{2}{c}{Toten (eV)} & Enthalpy of formation & \multicolumn{2}{c}{Mag} \\ \midrule
\ch{Cr3Fe3Mn7Ni3Si32} & 6.6947      & 0.0040 & -11.9586      & 0.1375     & 0.0186     \\
\ch{Cr5Fe5Mn3Ni3Si32} & 6.6705      & 0.0030 & -11.1991      & 0.1127     & 0.0223     \\
\ch{Cr5Fe3Mn5Ni3Si32} & 6.6852      & 0.0041 & -10.5200      & 0.1375     & 0.0456     \\
\ch{Cr3Fe5Mn5Ni3Si32} & 6.6801      & 0.0036 & -12.6426      & 0.0937     & 0.0209     \\ \bottomrule
\end{tabular}
\caption{Mean and stadard deviation of the total energy and magnetic moment per atom, plus enthalpy of formation of the listed mean energies (\ch{FeSi2}).}
\end{table}

The first result of table .. we make notice of is that the stability, as evaluated by the enthalpy of formation can be increased beyond the eqvimolar composition. This is accomplished in two distinct permutations, one with increments to  manganese relative to the other TM, and the other by reduction of chromium. Moreover the two respective permutations lie on the opposite side of the magnetic scale. The large magnetic moment of the manganese rich permutation and the low magnetic moment in the chromium poor permutation is very much in line with the observations made in the previous section. Recalling that in the magnetic moment in the eqvimolar composition was largely attributed to manganese and chromium atoms in the lattice. Thus increments to manganese or reduction of chromium would following impact the magnetic moment as in the two permutations. For this reason, additionally the permutation \ch{Cr5Fe3Mn5Ni3Si32} where the nonmagnetic elements is reduced and the magnetic elements are increased ,is equally magnetic. We however find no direct relation between stability and magnetism as his particular permutation is the least stable. An important property of table 8.5 is that the listed values are the mean value of the observed property for 5 distinct SQSs of the same permutation. For example we notice that while the highest magnetic moment in the first permutation is associated with the most stable SQS (from total energy considerations). The least stable supercell show the highest magnetic moment in \ch{Cr5Fe3Mn5Ni3Si32}. 

The respective band gap of the permutations (with PBE) can be seen in table ... Compared to the previous case, we find most SQSs of the permutations to exhibit a half-metallic character. 

\begin{table}[H]
\centering
\begin{tabular}{@{}ccccc@{}}
\toprule
                                                     &   & Spin up (eV) & Spin down (eV) & Total (eV) \\ \midrule
\multicolumn{1}{c|}{\multirow{5}{*}{\textbf{\ch{Cr3Fe3Mn7Ni3Si32}}}}   & A & 0.3390                & 0                       & 0                   \\
\multicolumn{1}{c|}{}                                & B & 0.4745                & 0                       & 0                   \\
\multicolumn{1}{c|}{}                                & C & 0.1342                & 0                       & 0                   \\
\multicolumn{1}{c|}{}                                & D & 0.1950                & 0.0063                  & 0.0063              \\
\multicolumn{1}{c|}{}                                & E & 0.4211                & 0                       & 0                   \\ \midrule
\multicolumn{1}{c|}{\multirow{3}{*}{\textbf{\ch{Cr5Fe5Mn3Ni3Si32}}}} & C & 0.2103                & 0                       & 0                   \\
\multicolumn{1}{c|}{}                                & D & 0.0674                & 0.0413                  & 0.0372              \\
\multicolumn{1}{c|}{}                                & E & 0.3619                & 0                       & 0                   \\ \midrule
\multicolumn{1}{c|}{\multirow{5}{*}{\textbf{\ch{Cr5Fe3Mn5Ni3Si32}}}} & A & 0.2082                & 0                       & 0                   \\
\multicolumn{1}{c|}{}                                & B & 0.4053                & 0                       & 0                   \\
\multicolumn{1}{c|}{}                                & C & 0.4659                & 0                       & 0                   \\
\multicolumn{1}{c|}{}                                & D & 0.0843                & 0.0121                  & 0.0121              \\
\multicolumn{1}{c|}{}                                & E & 0.3008                & 0                       & 0                   \\ \midrule
\multicolumn{1}{c|}{\multirow{4}{*}{\textbf{\ch{Cr3Fe5Mn5Ni3Si32}}}} & A & 0.3922                & 0                       & 0                   \\
\multicolumn{1}{c|}{}                                & C & 0.1285                & 0                       & 0                   \\
\multicolumn{1}{c|}{}                                & D & 0.2595                & 0.1004                  & 1.004               \\
\multicolumn{1}{c|}{}                                & E & 0.3591                & 0.1003                  & 0.0848              \\ \bottomrule
\end{tabular}
\caption{Total and spin dependent band gap of 4 permutations of CFMN (fesi2) with PBE GGA calculation. The structures that are excluded from this list either failed in calculations, or does not show any band gap. \textbf{Remove C and E from Mn3, these contain defects and no gap in DOS.}}
\end{table}

From table ..  we see that likewise to the stability and magnetization also the band gap changes in the different directions. To some degree we find positive results of the band gap in each direction, but we see particularly that permutations rich in manganese provide very encouraging results. This is made clear from the fact that \ch{Cr3Fe3Mn7Ni3Si32}, \ch{Cr5Fe3Mn5Ni3Si32} and \ch{Cr3Fe5Mn5Ni3Si32} all include amounts of manganese higher than the eqvimolar composistion and all associated SQSs show at least strong half-metalic charachter or semiconducting. On the other side \ch{Cr5Fe5Mn3Ni3Si32} is the sole permutation with less manganse and correspondingly show the least sign of a band gap. Moreover the relative stability of the SQSs give further merit to the proposition. In the first permutation we find that the highest total energy is associated with SQS B, which as seen in table .. exhibit the largest spin up band gap of the particular permutation. Furthermore the two semiconducting SQSs in the last permutation is the two most stable arrangements. Reversely, in the manganese-poor permutation we find that the sole semiconducting SQS is the second least stable of that compound. Lastly, the opposite is the case is true in the third permutation. Despite the total energy not varying tremendously between SQSs of the same permutation, as seen by the standard deviation in table .., the continuing trend between stability and band gap is a promising result to include.

However, we could ask the question if the half-metal results is truly a good indication. In the previous section we learned that all though the SQSs of the eqvimolar system favored the spin up channel in terms of a band gap, the structures were still narrow semiconductors, and with relatively lesser amount of manganese. It's important to note however, that in this analysis we didn't just increase/reduce manganese, we simultaneously altered the other elements as well. Therefore we can not conclude that this or that is the best direction, but from exclusively the permutations we tested there is clear indication that manganese is related to the band gap in a positive way.  

In this segment of the project we scarcely applied the more advanced functionals SCAN and HSE06,  in part to both the uncertainties mentioned in the previous section and the computational cost of the methods. However we did perform such calculations (HSE06) to further investigate the nature of the listed semiconducting SQSs. Both the manganese rich and poor semiconductors are validated with the HSE06 functional and find wider band gaps of 0.17 eV (0.57 and 0.26 in up and down) for the manganese rich composition, and 0.22 eV (0.77 eV in spin up) for the manganese poor composition. On the opposite side, the very narrow band gap of SQS D in the third permutation vanishes with HSE06 calculations. For the two stable semiconductors found in the reduced chromium permutation, simulations with the HSE06 functional resulted in a half-metal gap in spin up of 0.53 eV for SQS D, and a total band-gap of 0.27 eV for E, where the spin-up gap is 0.73 eV wide. 

In future research, it would be interesting to deliberately alter specific combinations from the results of table 8.6, for example reducing chromium and increasing manganese simultaneously.   



\chapter{Permutations of \ch{(CrFeMnNi)Si2}}
\label{sec:permutations}

Up until this point we have looked in detail at the high-entropy silicide \ch{(CrFeMnNi)Si2} and associated SQSs. However these structures are just the center of a larger quasi-ternary phase diagram consisting of the different possible distributions of elements Thus there exists many more permutations of this particular composition of a high-entropy silicide. In this section, we aim to expand our search of this diagram by generating SQSs slightly away from eqvimolar distribution of 3d elements. In table (bellow) we list the mean total energy and magnetic moment per atom with standard deviation and the enthalpy of formation of 4 permutations of the \ch{(CrFeMnNi)Si2} alloy. Ideally the permutations would differ only by one element, but the TDEP implementation insist in also reducing Nickel to stay consistent with the 48 atom supercell. 

\begin{table}[h!]
%\centering
\hskip-2.5cm\begin{tabular}{@{}cccccc@{}}
\toprule
       & \multicolumn{2}{c}{Total energy/atom (eV)} & Enthalpy of formation (eV) & \multicolumn{2}{c}{Final magnetic moment ($\mu_B$)} \\ \midrule
\ch{Cr3Fe3Mn7Ni3Si32} & - 6.6947  & 0.0040 & -11.9586  & 0.1375  & 0.0186     \\
\ch{Cr5Fe5Mn3Ni3Si32} & - 6.6705  & 0.0030 & -11.1991  & 0.1127  & 0.0223     \\
\ch{Cr5Fe3Mn5Ni3Si32} & - 6.6852  & 0.0041 & -10.5200  & 0.1375  & 0.0456     \\
\ch{Cr3Fe5Mn5Ni3Si32} & - 6.6801  & 0.0036 & -12.6426  & 0.0937  & 0.0209     \\
\ch{Cr3Fe3Mn3Ni7Si32} & - 6.3921  & 0.0078 & -10.9614  & 0.0159  & 0.0101 \\ \bottomrule
\end{tabular}
\caption{Mean and stadard deviation of the total energy and magnetic moment per atom, plus enthalpy of formation of the listed mean energies (\ch{FeSi2}).}
\end{table}

The first result of table .. we make notice of is that the stability, as evaluated by the enthalpy of formation can be increased beyond the eqvimolar composition. This is accomplished in two distinct permutations, one with increments to  manganese relative to the other TM, and the other by reduction of chromium. Moreover the two respective permutations lie on the opposite side of the magnetic scale. The large magnetic moment of the manganese rich permutation and the low magnetic moment in the chromium poor permutation is very much in line with the observations made in the previous section. Recalling that in the magnetic moment in the eqvimolar composition was largely attributed to manganese and chromium atoms in the lattice. Thus increments to manganese or reduction of chromium would following impact the magnetic moment as in the two permutations. For this reason, additionally the permutation \ch{Cr5Fe3Mn5Ni3Si32} where the nonmagnetic elements is reduced and the magnetic elements are increased ,is equally magnetic. We however find no direct relation between stability and magnetism as his particular permutation is the least stable. An important property of table 8.5 is that the listed values are the mean value of the observed property for 5 distinct SQSs of the same permutation. For example we notice that while the highest magnetic moment in the first permutation is associated with the most stable SQS (from total energy considerations). The least stable supercell show the highest magnetic moment in \ch{Cr5Fe3Mn5Ni3Si32}. 

The respective band gap of the permutations (with PBE) can be seen in table ... Compared to the previous case, we find most SQSs of the permutations to exhibit a half-metallic character. 

\begin{table}[H]
\centering
\begin{tabular}{@{}ccccc@{}}
\toprule
                                                     &   & Spin up (eV) & Spin down (eV) & Total (eV) \\ \midrule
\multicolumn{1}{c|}{\multirow{5}{*}{\textbf{\ch{Cr3Fe3Mn7Ni3Si32}}}}   & A & 0.3390                & 0                       & 0                   \\
\multicolumn{1}{c|}{}                                & B & 0.4745                & 0                       & 0                   \\
\multicolumn{1}{c|}{}                                & C & 0.1342                & 0                       & 0                   \\
\multicolumn{1}{c|}{}                                & D & 0.1950                & 0.0063                  & 0.0063              \\
\multicolumn{1}{c|}{}                                & E & 0.4211                & 0                       & 0                   \\ \midrule
\multicolumn{1}{c|}{\multirow{1}{*}{\textbf{\ch{Cr5Fe5Mn3Ni3Si32}}}}   & D & 0.0674                & 0.0413                  & 0.0372              \\ \midrule
\multicolumn{1}{c|}{\multirow{5}{*}{\textbf{\ch{Cr5Fe3Mn5Ni3Si32}}}} & A & 0.2082                & 0                       & 0                   \\
\multicolumn{1}{c|}{}                                & B & 0.4053                & 0                       & 0                   \\
\multicolumn{1}{c|}{}                                & C & 0.4659                & 0                       & 0                   \\
\multicolumn{1}{c|}{}                                & D & 0.0843                & 0.0121                  & 0.0121              \\
\multicolumn{1}{c|}{}                                & E & 0.3008                & 0                       & 0                   \\ \midrule
\multicolumn{1}{c|}{\multirow{4}{*}{\textbf{\ch{Cr3Fe5Mn5Ni3Si32}}}} & A & 0.3922                & 0                       & 0                   \\
\multicolumn{1}{c|}{}                                & C & 0.1285                & 0                       & 0                   \\
\multicolumn{1}{c|}{}                                & D & 0.2595                & 0.1004                  & 1.004               \\
\multicolumn{1}{c|}{}                                & E & 0.3591                & 0.1003                  & 0.0848              \\ \midrule
\multicolumn{1}{c|}{\multirow{1}{*}{\textbf{\ch{Cr3Fe3Mn3Ni7Si32}}}} & - & -                & -                      & -                   \\ \bottomrule
\end{tabular}
\caption{Total and spin dependent band gap of 4 permutations of CFMN (fesi2) with PBE GGA calculation. The structures that are excluded from this list either failed in calculations, or does not show any band gap.<}
\end{table}

From table ..  we see that likewise to the stability and magnetization also the band gap changes in the different directions. To some degree we find positive results of the band gap in each direction, but we see particularly that permutations rich in manganese provide very encouraging results. This is made clear from the fact that \ch{Cr3Fe3Mn7Ni3Si32}, \ch{Cr5Fe3Mn5Ni3Si32} and \ch{Cr3Fe5Mn5Ni3Si32} all include amounts of manganese higher than the eqvimolar composistion and all associated SQSs show at least strong half-metalic charachter or semiconducting. On the other side \ch{Cr5Fe5Mn3Ni3Si32} is the sole permutation with less manganse and correspondingly show the least sign of a band gap. Moreover the relative stability of the SQSs give further merit to the proposition. In the first permutation we find that the highest total energy is associated with SQS B, which as seen in table .. exhibit the largest spin up band gap of the particular permutation. Furthermore the two semiconducting SQSs in the last permutation is the two most stable arrangements. Reversely, in the manganese-poor permutation we find that the sole semiconducting SQS is the second least stable of that compound. Lastly, the opposite is the case is true in the third permutation. Despite the total energy not varying tremendously between SQSs of the same permutation, as seen by the standard deviation in table .., the continuing trend between stability and band gap is a promising result to report.

In figure 8.12 we plot the projected density of states around $E_F$ of the 4 permutations. Note that away from the Fermi energy the projected density of states is analogous to the parent compound, see appendix ..

\begin{figure}[H]
\includegraphics[width=\linewidth]{results/fesi2/permutations/perm_LDOS_crop.jpg}
\caption{Projected density of states of (a) \ch{Cr3Fe3Mn7Ni3Si32} (SQS B), (b) \ch{Cr5Fe5Mn3Ni3Si32} (SQS C), (c) \ch{Cr5Fe3Mn5Ni3Si32} (SQS A), (d) \ch{Cr3Fe5Mn5Ni3Si32} (SQS D)}
\end{figure}

The above figures is based on the most stable SQS in each permutation, as will the analysis. Thus the features of these figures does not necessarily represent the complete set of SQSs of the permutation. As experienced in table 8.6 and previous examples in this project, the band gap and properties of each permutation can vary between SQSs. But given that the structures came out on top in terms of total energy means that they are the most probable configuration of the real alloy, hence also the features of that supercell. 

With that said, the plotted PDOSs in figure 8.12 clearly illustrate the characteristics shown in table 8.6. We see clear indication of a spin up band gap in \ch{Cr3Fe3Mn7Ni3Si32} (a) and \ch{Cr5Fe3Mn5Ni3Si32} (c), and a total band gap in \ch{Cr3Fe5Mn5Ni3Si32} (d). Not so clear is that the density of states of \ch{Cr5Fe5Mn3Ni3Si32} (b) contain very small nonzero values at $E_F$ and the unoccupied states is shifted very slightly above the Fermi energy, prohibiting an otherwise clear band gap. In figure 8.13 the total density of states of SQS D and E of this permutation is shown, the above point is seen also in SQS E, where the "band gap" is shifted above $E_F$.
 
\begin{figure}[H]
	\begin{subfigure}{.5\textwidth}
		\includegraphics[width=\textwidth]{results/fesi2/permutations/mnni3_DOS.png}
		\caption{SQS D}
	\end{subfigure}
	\begin{subfigure}{.5\textwidth}
		\includegraphics[width=\textwidth]{results/fesi2/permutations/mnni3_DOS_E.png}
		\caption{SQS E}
	\end{subfigure}
	\caption{Density of states around $E_F$ of SQS D and E \ch{Cr5Fe5Mn3Ni3Si32}}
\end{figure}


In figure 8.6 we saw that electrons from manganese atoms in particular was a key contributor as to why the spin down channel of \ch{(CrFeMnNi)Si2} was metallic in the stable supercell D. This is also largely the case in the permutations shown above in figure 8.12.    
 
The proportion of manganese atoms in the alloy seems to offer a very positive effect on the band gap in spin up, but is often detrimental to spin down. This is seen in figure 8.12 (a) and (c) for \ch{Cr3Fe3Mn7Ni3Si32} and \ch{Cr5Fe3Mn5Ni3Si32} respectively, that both contain increased amounts of manganese. By reducing the number of Mn as in (b) we still find that the Mn electrons plague the states at $E_F$ in spin down. In analog we see from (b) and (c) that also Cr negatively impacts to the band gap especially in spin up. The sole permutation with clear evidence of a spin down gap is from the chromium poor permutation plotted in (d). Also in this structure we see that the effects of Mn around $E_F$ is dampened in comparison to the other permutations, despite containing relatively increased amounts of Mn to the eqvimolar alloy.  

\begin{figure}[H]
	\centering
	\includegraphics[width=.6\textwidth]{results/fesi2/permutations/ni7_PDOS.png}
	\caption{Projected density of states of \ch{Cr3Fe3Mn3Ni7Si32} around $E_F$}
\end{figure}

An important property of these results is that because each permutation alters simultaneous elements, interpreting and relating the results to a particular alteration is challenging. For example, is the result of the \ch{Cr5Fe3Mn5Ni3Si32} permutation a consequence of less Fe or increments to both Cr and Mn? Furthermore is the exaggerated band gap in spin up of \ch{Cr3Fe3Mn7Ni3Si32} a product of increasing manganese or reducing the other elements. From the comparatively large gaps in spin up of \ch{Cr3Fe3Mn7Ni3Si32} and \ch{Cr3Fe5Mn5Ni3Si32} and the more present Cr states in spin up in the Cr rich permutations we here conclude that the band gap is related to lessening of chromium, more so than other effects. Despite of this we generally find positive results regarding most of the permutations as seen in table 8.6, the exception to this \ch{Cr3Fe3Mn3Ni3si32}. This particular permutation in opposition to the others in this section increases the proportion of Ni at the cost of the other 3d elements. The projected density of states is displayed in figure [REF]. From both this structure, but also the PDOS of \ch{Cr3Fe3Mn7Ni3Si32} we see that reducing chromium does not always improve the band gap. It's clear that the \ch{Cr3Fe5Mn5Ni3Si32} alloy manage to strike a balance of distribution that results in a specific interplay between the 3d elements. For this reason we more closely investigate the properties of this structure, the probability distribution of SQS D (highest stability) is plotted bellow in figure 8.15.

\begin{figure}[H]
	\centering
	\includegraphics[width=\textwidth]{results/fesi2/permutations/_crni3_D_PDF.png}
	\caption{Probability distribution functions to \ch{Cr3Fe5Mn5Ni3Si32} SQS D, \textbf{Maybe make larger}}
\end{figure}

\textbf{Comment figure \\}

\textbf{Add some figures from HSE06?}
In this segment of the project we scarcely applied the more advanced functionals SCAN and HSE06, in part to both the uncertainties mentioned in the previous section and the computational cost of the methods. However we did perform such calculations (HSE06) to further investigate the nature of the listed semiconducting SQSs. Both the manganese rich and poor semiconductors are validated with the HSE06 functional and find wider band gaps of 0.17 eV (0.57 and 0.26 in up and down) in \ch{Cr3Fe3Mn7Ni3Si32} SQS D, and 0.22 eV (0.77 eV in spin up) in \ch{Cr5Fe5Mn3Ni3Si32} SQS D. On the opposite side, the very narrow band gap in \ch{Cr5Fe3Mn5Ni3Si32} vanishes with HSE06 calculations. For the two stable semiconductors found in \ch{Cr3Fe5Mn5Ni3Si32}, simulations with the HSE06 functional resulted in a half-metal with a spin up of 0.53 eV for SQS D, and a total band-gap of 0.27 eV for E, where the spin-up gap is 0.73 eV wide
. \textbf{Comparing to table 8.6 we observe that ..}. \textbf{As for the SCAN functional ...} 

\textbf{Conclusion this section \\}

\section{New compositions}

In similar fashion to the previous sections, we here begin by presenting the mean and standard deviation of the total energy and magnetization of a set of SQSs corresponding to different high-entropy silicides of the \ch{Fesi2} unit cell. The compositions we have tested are deliberate combinations intended to investigate both the impact of manganese by replacing the element with Co or Ti, and concepts related to HEA theory such as the atomic size effect. Furthermore Co is a very common element in many stable HEA, as seen in section 2.2, thus we include 3 compositions with Co to study the impact on stability and the functional properties. The results of the aforementioned alloys can be seen bellow in table 9.1, note that all compounds contain a total of 48 atoms as before.  

\begin{table}[H]
\centering
\begin{tabular}{@{}cccccc@{}}
\toprule
Composition           & \multicolumn{2}{c}{\begin{tabular}[c]{@{}c@{}}Toten \\ (eV)\end{tabular}} & \multicolumn{2}{c}{\begin{tabular}[c]{@{}c@{}}Mag\\ ($\mu_B$)\end{tabular}} & \begin{tabular}[c]{@{}c@{}}$\Delta H$\\ (eV)\end{tabular} \\ \midrule
                      & mean                                 & std                                & mean                                 & std                                  & mean                                                      \\ \midrule
\ch{Cr4Fe4Co4Ni4Si32} & - 6.4655                             & 0.0056                             & 0.0083                               & 0.0155                               & - 12.7536                                                 \\
\ch{Co4Fe4Mn4Ni4Si32} & - 6.4731                             & 0.0046                             & 0.0000                               & 0.0000                               & - 15.0836                                                 \\
\ch{Cr4Fe4Ti4Ni4Si32} & - 6.4217                             & 0.0087                             & 0.0305                               & 0.0293                               & - 7.5040                                                  \\
\ch{Cr4Fe4Mn4Ti4Si32} & -6.6994                              & 0.0071                             & 0.1142                               & 0.0641                               & - 7.3060                                                  \\
\ch{Cr4Fe4Mn4Co4Si32} & -6.7687                              & 0.0034                             & 0.1331                               & 0.0326                               & - 13.7796                                                 \\ \bottomrule
\end{tabular}
\caption{Overview new compositions}
\end{table}

From table 9.1 we see that the stability of the relative compositions vary greatly. By introducing cobalt to the alloys, particularly at the cost of manganese result in a large positive effect on the stability, contrary replacing either manganese or nickel with titanium significantly lowers the stability. \textbf{Wait for new formation enthalpies.} In table 9.1 we have listed the mean magnetic moment of the compositions, in line with previous results in this project the magnetization is very dependent on chromium and manganese. This is seen by the overall lowest magnetic moments in the two compositions without these elements, and reversely the highest magnetic moments is found for compositions with both Cr and Mn. Comparing the magnetic moment of \ch{(CrFeCoNi)Si2} and \ch{(CoFeMnNi)Si2} it seems in our study that chromium is most responsible for the magnetic moment in these alloys. Furthermore we find that substituting Ni with both Ti and Co result in more magnetic compounds. These are truly surprising results, one would expect that the magnetic moments would be larger in the ferromagnetic elements Ni, Fe and Co than Cr, Mn and ti. This could go back to our simplistic and superficial study of the magnetic properties in this project, additionly the PBE functional as we covered in section .. have shown limitations for 3d elements and particularly Ni. Thus this could be a factor affecting our results. Another factor is that we here based our comparison on the mean values between 5 SQSs. As we have experienced throughout this project the unieqeness of the SQSs can be troublesome to handle, and our best guess is to study the most stable super-cell. Bellow in table 9.2 we list the magnetic moments of the most stable SQSs. Here we find several dissimilarities to the mean value such as the \ch{Cr4Fe4Co4Ni4Si32} being nonmagnetic in the most stable supercell. Thus based on the utmost stable configurations we can state that replacing either Cr or Mn (with Co) removes the magnetic moment in the alloy. Furthermore we find from these supercells that the magnetic moment is reduced by replacing Ni with Ti, and increased from Co. These results are in much better accordance with previous knowledge of ferromagnetic elements and their interplay in high-entropy alloys.

\begin{table}[H]
\centering
\begin{tabular}{@{}lc@{}}
\toprule
\multicolumn{1}{c}{Composition} & \begin{tabular}[c]{@{}c@{}}Magnetic moment\\ ($\mu_B$)\end{tabular} \\ \midrule
\ch{Cr4Fe4Co4Ni4Si32}           & 0                                                                   \\
\ch{Co4Fe4Mn4Ni4Si32}           & 0                                                                   \\
\ch{Cr4Fe4Ti4Ni4Si32}           & 0,0653                                                              \\
\ch{Cr4Fe4Mn4Ti4Si32}           & 0,0785                                                              \\
\ch{Cr4Fe4Mn4Co4Si32}           & 0,1666                                                              \\ \bottomrule
\end{tabular}
\caption{Final magnetic moment of the most stable supercell of each composition.}
\end{table}

In regards to the band gap of these compositions, we find most to be metals. The band gap of the most stable SQS of each composition is listed in table 4.3, where we calculate the band gap from the eigenvalues at different occupancy cutoffs. As before the 0 band-gap is caused by defect states in the band gap. By increasing the criteria, in other words only consider states with occupancy above a certain threshold, the band gap become finite at $occ = 0.1$ and converge to around $0.02 - 0.06$ eV depending on composition, when only considering full/empty states.  

\begin{table}[H]
\centering
\begin{tabular}{@{}ccccc@{}}
\toprule
\multicolumn{1}{l}{Composition}                   & $occ$                     & \begin{tabular}[c]{@{}c@{}}$E_\text{G} ^\text{up, eigen}$\\ (eV)\end{tabular} & \begin{tabular}[c]{@{}c@{}}$E_\text{G} ^\text{dw, eigen}$\\ (eV)\end{tabular} & \begin{tabular}[c]{@{}c@{}}$E_\text{G} ^\text{tot, eigen}$\\ (eV)\end{tabular} \\ \midrule
\multicolumn{1}{c|}{\multirow{3}{*}{\ch{CrFeCoNiSi2}}}                  & \multicolumn{1}{c|}{0.5}  & 0                                                                             & 0                                                                             & 0                                                                              \\
\multicolumn{1}{c|}{}                             & \multicolumn{1}{c|}{0.1}  & 0.00095                                                                       & 0.0399                                                                        & 0.00095                                                                        \\
\multicolumn{1}{c|}{}                             & \multicolumn{1}{c|}{0.01} & 0.063                                                                         & 0.063                                                                         & 0.063                                                                          \\ \midrule
\multicolumn{1}{c|}{\multirow{3}{*}{\ch{CrFeTiNiSi2}}} & \multicolumn{1}{c|}{0.5}  & 0.0067                                                                        & 0                                                                             & 0                                                                              \\
\multicolumn{1}{c|}{}                             & \multicolumn{1}{c|}{0.1}  & 0.061                                                                         & 0.0087                                                                        & 0.0087                                                                         \\
\multicolumn{1}{c|}{}                             & \multicolumn{1}{c|}{0.01} & 0.061                                                                         & 0.037                                                                         & 0.037                                                                          \\ \midrule
\multicolumn{1}{c|}{\multirow{3}{*}{\ch{CoFeMnNiSi2}}} & \multicolumn{1}{c|}{0.5}  & 0                                                                             & 0                                                                             & 0                                                                              \\
\multicolumn{1}{c|}{}                             & \multicolumn{1}{c|}{0.1}  & 0.0037                                                                        & 0.0037                                                                        & 0.0037                                                                         \\
\multicolumn{1}{c|}{}                             & \multicolumn{1}{c|}{0.01} & 0.0268                                                                        & 0.0268                                                                        & 0.0268                                                                         \\ \midrule
\multicolumn{1}{c|}{\multirow{3}{*}{\ch{CrFeMnTiSi2}}} & \multicolumn{1}{c|}{0.5}  & 0                                                                             & 0                                                                             & 0                                                                              \\
\multicolumn{1}{c|}{}                             & \multicolumn{1}{c|}{0.1}  & 0.021                                                                         & 0.00049                                                                       & 0                                                                              \\
\multicolumn{1}{c|}{}                             & \multicolumn{1}{c|}{0.01} & 0.03                                                                          & 0.03                                                                          & 0.022                                                                          \\ \midrule
\multicolumn{1}{c|}{\multirow{3}{*}{\ch{CrFeMnCoSi2}}} & \multicolumn{1}{c|}{0.5}  & 0.461                                                                         & 0                                                                             & 0                                                                              \\
\multicolumn{1}{c|}{}                             & \multicolumn{1}{c|}{0.1}  & 0.607                                                                         & 0.0218                                                                        & 0.0218                                                                         \\
\multicolumn{1}{c|}{}                             & \multicolumn{1}{c|}{0.01}                      & 0.607                                                                         & 0.0245                                                                        & 0.0245                                                                         \\ \bottomrule
\end{tabular}
\caption{Band gaps of the most stable SQS of $\beta-$ \ch{FeSi2} high-entropy silicide compositions as a function of occupancy in the eigenvalues.}
\end{table}

The one exception to the metallic compositions is the \ch{CrFeMnCoSi2} composition with a gap of around 0.5 eV in the spin up channel. As seen in previous cases, this structure despite of the large band gap contains also a small amount of defect states. This is seen in the projected density of states plotted in figure 4.1, where we observe small nonzero values of the density of states at the Fermi energy. The PDOSs of the other compositions is found in appendix ..  
  
\begin{figure}[H]
\centering
\includegraphics[width=\textwidth]{results/fesi2/composistions/crfemnco_PDOS.png}
\caption{Projected density of states of \ch{(CrFeMnCo)Si2}.}
\end{figure}

Contrary to the above case, inserting Co in the place of manganese clearly result in a metallic structure, as seen in the density of states in figure 9.2 a. Replacing Mn with Ti instead we recall from table 9.3 a a very small defect band gap in spin up, however from figure 9.2b we observe that $E_\text{G}^{dos}$ is equal to zero, thus $E_G ^\text{dos} \neq E_G ^{eigen}$. Comparing the density of states of \ch{(CrFeCoNi)Si2} and \ch{(CrFeTiNi)Si2} tha the latter is magnetic and the former nonmagnetic, as we discussed previously.  


\begin{figure}[H]
\begin{subfigure}{.5\textwidth}
\includegraphics[width=\textwidth]{results/fesi2/composistions/crfeconi_DOS.png}
\caption{\ch{(CrFeCoNi)Si2}}
\end{subfigure}
\begin{subfigure}{.5\textwidth}
\includegraphics[width=\textwidth]{results/fesi2/composistions/crfetini_DOS.png}
\caption{\ch{(CrFeTiNi)Si2}}
\end{subfigure}
\caption{Density of states of a) \ch{(CrFeCoNi)Si2} and b) \ch{(CrFeTiNi)Si2}.}
\end{figure} 

Above we have looked at the band gap of the most stable SQS of each composition, but as we have experienced in other cases in this project, the properties can vary between SQSs of the same composition. In both \ch{CrFeCoNiSi2} and \ch{CrFeMnTiSi2} we found only metalic supercells with the exception of one SQS in the latter with a very small defect band gap in spin up. Similarly small defect band gaps was observed in two SQSs of \ch{CrFeTiNiSi2} and the rest as metals. In \ch{CrFeMnCoSi2} we found a large defect band gap in spin in the most stable configuration, here we find similar band gaps in two other SQSs as well, and two structures. The most interesting case was found in \ch{CoFeMnNiSi2} where we observed small total band gaps without defect states in two SQSs, these are seen in figure 9.3. In agreement with the nonmagnetic character of this composition, the DOS is symmetric with respect to spins. 

\begin{figure}[H]
\begin{subfigure}{.5\textwidth}
\includegraphics[width=\textwidth]{results/fesi2/composistions/cofemnni_E_DOS.png}
\caption{SQS 1}
\end{subfigure}
\begin{subfigure}{.5\textwidth}
\includegraphics[width=\textwidth]{results/fesi2/composistions/cofemnni_E_DOS.png}
\caption{SQS 2}
\end{subfigure}
\caption{Density of states of two SQSs of \ch{(CoFeMnNi)Si2}.}
\end{figure} 

Thus we find examples where the most stable SQS is representative of the set, and other cases where we observe meaningful distinctions between the most stable and other possible configurations. Because of limited time examining properties and options affecting both the magnetism and stability of the alloys and respective supercells, it's important to consider the different configurations as a a broader search would not necessarily yield the same relationship. And to show some of the limitations that apply when using the special quasi-random structures method. \textbf{Can I say this?}. It appears that we find limited success overall, but particularly when substituting either Cr or Mn as outlined in the previous section. Furthermore Ti is not as successful as Co. \textbf{Continue conclusion this and entire project}
 

 


\part{Conclusion}
Write conclusion here

\appendix

\chapter{Figures}
\label{appendix:equi}

\section{Density of states}

\begin{figure}[H]
	\centering
	\includegraphics[width=\textwidth]{results/fesi2/E_TDOS.png}
	\caption{Density of states [states/eV] of SQS E of \ch{(CrFeMnNi)Si2}.}
\end{figure}

\begin{figure}[H]
	\centering
	\includegraphics[width=\textwidth]{results/fesi2/C_DOS.png}
	\caption{Density of states [states/eV] of SQS C \ch{(CrFeMnNi)Si2}. NEDOS represents the number of points in the DOS calculation.}
\end{figure}

\begin{figure}[H]
\includegraphics[width=\textwidth]{results/fesi2/permutations/mnni3_DOS_E.png}
\caption{Density of states [states/eV] of SQS C of \ch{Cr5Fe5Mn3Ni3Si32}, illustrating the small finite DOS at $E_F$ due to the impurity gap.}
\end{figure}

\section{Projected density of states}

\begin{figure}[H]
\includegraphics[width=\textwidth]{results/fesi2/A_PDOS.png}
\caption{Projected density of states [states/eV] of SQS A of \ch{(CrFeMnNi)Si2}}
\end{figure}

\begin{figure}[H]
\includegraphics[width=\textwidth]{results/fesi2/B_PDOS.png}
\caption{Projected density of states [states/eV] of SQS B of \ch{(CrFeMnNi)Si2}}
\end{figure}

\begin{figure}[H]
\includegraphics[width=\textwidth]{results/fesi2/E_PDOS.png}
\caption{Projected density of states [states/eV] of SQS E of \ch{(CrFeMnNi)Si2}}
\end{figure}

\section{Charge density}

\begin{figure}[H]
\includegraphics[width=\textwidth]{results/fesi2/C_CHGCAR.jpg}
\caption{Contour plot of the charge density of SQS C of \ch{(CrFeMnNi)Si2}.}
\end{figure}


\chapter{PDFs}
\label{appendix:pdf}
something
\chapter{Charge density}\label{appendix: chg}

\backmatter{}
\printbibliography
\end{document}
