\chapter{Density-Functional Theory}
\label{sec:DFT}

\section{Review of Quantum Mechanics}

\subsection{The Shr\"{o}dinger equation}
The Schr\"{o}dinger equation composed of the wavefunction $\Psi(\vec{r},t)$ and Hamiltonian $\hat{H}(\vec{r},t)$ where $\vec{r}$ and $t$ is the spatial position and time respectfully.
\begin{equation}
    i\hbar\frac{\partial}{\partial t}\Psi(\vec{r}, t) = \hat{H}(\vec{r},t)\Psi(\vec{r}, t)
\end{equation}
The time-independent shr\"{0}dinger equation for the eigenvalues $E_k$ of the $k$-th eigenvalue $\psi_k(\vec{r}$
\begin{equation}
    \hat{H}\psi_k(\vec{r}) = E_k \psi_k(\vec{r})
\end{equation}
Extending to a system comprised of multiple particles, we have the many-particle Shr\"{o}dinger equation, involving the many-body Hamiltonian. This quantity is composed of the kinetic energy of $N_e$ electrons $T_e$, the interaction energy between electrons $U_{ee}$, the kinetic energy of $N_n$ nuclei, the coulomb interaction between nuclei $U_{nn}$, and finally the attractive interaction between nuclei and electrons $U_{en}$. In the equation bellow for the may-body equation, we use the following symbols and notation: $m_e$ = electron mass, $m_n$ = nuclei mass, $\epsilon_0$ = permittivity in vacuum, $q$ = particle charge, $\alpha$ = nuclei number, $Z_a$ = atom number of nuclei $\alpha$, $r$ = position of electron, $R$ = position of nuclei.
\begin{align}
    \hat{H} &= T_e + T_n + U_{ee} + U_{nn} + U_{en} \\
    \begin{split}
        &= -\sum_{j=1}^{N_e}\frac{\hbar^2\nabla_j}{2m_e} - \sum_{\alpha=1}^{N_n}\frac{\hbar^2 \nabla_\alpha}{2m_n} + \frac{1}{4\pi \epsilon_0}\sum_{j=1}^{N_e}\sum_{j'<j}\frac{q^2}{|r_j - r_{j'}|} \\
    &+ \frac{1}{4\pi\epsilon_0}\sum_{\alpha=1}^{N_n}\sum_{\alpha' < \alpha}\frac{q^2Z_\alpha Z_{\alpha'}}{R_\alpha - R_{\alpha'}} - \frac{1}{4\pi\epsilon_0}\sum_{j=1}^{N_e}\sum_{\alpha=1}^{N_n}\frac{q^2Z_\alpha}{|r_j - R_\alpha|}
    \end{split}
\end{align} 

\subsection{Simplifications and approximations to solve the many-electron Shr\"{o}dinger equation}

\subsubsection{Born-Oppenheimer}
Challenges with solving many-particle Shr\"{o}dinger equation is i) computationally expensive, ii) need to know how $\Psi$ depends on single particle wavefunctions $\psi_k$. To solve this complex problem, we need approximations. Particularly Born-Oppenheimer and Harte-Fock approximations. The first makes the cleaver and reasonable assumption that since the electron mass is negligibly small in comparison to that of a nuclei, we can treat the nuclei as point charges, enabling us to divide the eigenfunction into a separate electronic and nuclear part, ie
\begin{equation}
    \Psi_{k}^{en}(\vec{r}, \vec{R}) \approx \Psi_k(\vec{r}, \vec{R}) \Theta_k(\vec{R})
\end{equation}
where we have written the complete wavefunction in terms of an electronic part $\Psi_k(\vec{r}, \vec{R})$ and nuclear part $\Theta_k(\vec{R})$. The dependencies come from the fact that electrons can respond instantaneously to new positions of the nuclei, therefore the $\vec{R}$ dependence. Writing this in terms of the Hamiltonian we get
\begin{align}
    &\left( T_{e} + U_{ee} + U_{en} \right) \Psi_k(\vec{r}, \vec{R}) = E_k(\vec{R})\Psi_k(\vec{r}, \vec{R}) \\
    &\left( T_{n} + U_{nn} + E_k(\vec{R}) \right) \Theta(\vec{R}) = E_{k}^{en}(\vec{R})\Theta_k(\vec{r}, \vec{R}).
\end{align}
The two sections are interrelated through the electronic energy eigenvalue $E_k(\vec{R})$. Furthermore, the left hand side of the nuclear part can be simplified to $U_{nn} + E_k(\vec{R})$, assuming that the kinetic energy of point charges is zero. This simplified expression for the nuclear left hand side is called for the potential energy surface (EPS).

\subsubsection{Hartree-Fock}
The next step in line is to find a wavefunction that can describe all electrons in a system. This was originally done by Hartree, which assumed that electrons can be described independently and suggested the ansatz for a two-electron wavefunction
\begin{equation}
    \Psi_k(\vec{r_1}, \vec{r_2}) = A \cdot \psi_1(\vec{r_1}) \psi_2(\vec{r_2}),
\end{equation}
where A is the normalization constant. However this approximation does not account for the fact that electrons are indistinguishable and hence does not obey the Pauli exclusion principle for ferminons. This was overcome with the Hartree-Fock approximation that implement an anti-symmetric wavefunction. The full expression is given bellow
\begin{equation}
    \Psi_k(\vec{r_1}, \vec{r_2}) = \frac{1}{\sqrt{2}} \Big( \psi_1(\vec{r_1}) \psi_2(\vec{r_2}) - \psi_1(\vec{r_2})\psi_2(\vec{r_1}) \Big)
\end{equation}

The Hartree-Fock (HF) approximation makes the electrons distinguishable and hence obey the Pauli exclusion principle, this means that the exchange energy is accounted for. On the other side, HF is not a complete description as it fails to model the electron correlations. 

\subsubsection{The Variational principle}
In materials science, the overarching concern is the ground-state properties of a system. This can be found efficiently and easy by whats known as the variational principle. This states that the energy of any trial wavefunction will always be higher than the ground-state energy $E_0$, ie
\begin{equation}
    E_0 = \langle\psi_0|H|\psi_0\rangle \leq \langle\psi|H|\psi\rangle = E
\end{equation}
This enable us to find the ground state energy and corresponding wavefunction by a minimization technique. Next, we will present the basics of the density functional theory for how these equations can be solved numerically and efficiently in order to study real materials and systems.

\section{Fundamentals of Density-Functional Theory}
The density functional theory was developed by Hohenberg and Kohn in 1964 and revolved around the fact that the ground-state density can be expressed in terms of the ground-state wavefunction. We have
\begin{equation}
    n_0(r) = |\Psi_0(r)|,
\end{equation}
furthermore the theorem states that all ground-state physical properties can be found as unique functionals of the ground-state density. The biggest upside of this, is that instead of trying to solve the many-body Shr\"{o}dinger equation to obtain the ground-state wavefunction, we have reduced the computational complexity from $3N_e$ to $3$. Thus, the Hohenberg and Kohn density functional theory makes for a promising and effective method to obtain the ground-state properties of a system, given that the exact electron density functional is known. However, this is still 60 years later unknown. 

The density functional theory build on two specific theories, called the Hohenberg-Kohn theorems. They are:
\begin{enumerate}
    \item For any system of interacting particles in an external potential $V_{ext}$, the density is uniquely determined.
    \item There exists a variational principle for the energy density functional such that, if $n$ is not the electron density of the ground-state, them $E[n_0] < E[n]$.
\end{enumerate}
The proof behind both theorems can be found in appendix .. A direct result of the second theorem is the energy can be described as a function of the density
\begin{equation}
    E[n] = T[n] + U_{ee}[n] + U_{en}[n],
\end{equation}
where the first two terms $T[n]$ and $U_{ee}[n]$ make up the Hohenberg-Kohn functional. 

We know move on to the Kohn-Sham equations, in which Kohn and Sham expressed the exact ground-state density from Hartree type wavefunctions. 
\begin{equation}
    \Psi(\vec{r}_1, \vec{r}_2 , .., \vec{r}{N_e}) = \psi_1^{KS}(\vec{r}_1)\psi_2^{KS}(\vec{r}_2)...\psi_{N_e}^{KS}(\vec{r}_{N_e})
\end{equation}
In which, $\psi_j^{KS}$ are auxiliary independent single-particle wavefunctions. We know modify the equation for total energy as a function of density defined by the second theorem, to include the single auxiliary wavefunctions and their corresponding kinetic energy and interaction energy. We get:
\begin{equation}
    E[n] = T_s[n] + U_s[n] + U_{en}[n] + (T[n] - T_s[n]) + (U_{ee}[n] - U_s[n]).
\end{equation}
with the s subscript denoting the single particle wavefunctions. The latter two terms are known as the exchange-correlation energy $E_{xc}$
\begin{equation}
    E_{xc}[n] = \Delta T + \Delta U
\end{equation}

This term is responsible for the many-electron interaction. The complete total energy functional can now be expressed as
\begin{equation}
    \begin{split}
    E[n] &= \overbrace{\sum_j \int \psi_j^{KS*}\frac{-\hbar^2 \nabla^2}{2m}\psi_j^{KS}d\vec{r}}^{T_s[n]} + \overbrace{\frac{1}{2}\frac{1}{4\pi \epsilon_0} \int \int q^2 \frac{n(\vec{r})n(\vec{r}')}{|\vec{r} - \vec{r}'|}d\vec{r}\vec{r}'}^{U_s[n]} \\ 
        &+ \underbrace{\int V_{en}(\vec{r}n(\vec{r})d\vec{r}}_{U_en[n]} + \underbrace{(T[n] - T_s[n]) + (U_{ee}[n] - U_s[n])}_{E_{xc}[n]}
    \end{split}
\end{equation}

Finally we write the complete expression for the Kohn-Sham single-electron equations given an exact exchange-correlation energy and utilizing the variational principle described previously
\begin{equation}
    \bigg\{ -\frac{\hbar^2}{2m_e}\nabla^2_s + v_H(\vec{r}) + V_{en}(\vec{r}) + V_{xc}(\vec{r} \bigg\}\psi_s^{KS}(\vec{r}) = \epsilon_s^{KS}(\vec{r})\psi_s^{KS}(\vec{r}),
\end{equation}
Define $V_H$ and $V_{xc}$ and mention that the former include self interaction that can be accounted for in XC functional. Finally, the total energy of the many-electron system is defined as
\begin{equation}
    E[n] = \sum_j \epsilon_j^{KS} - \frac{1}{2}\frac{1}{4\pi\epsilon_0} \int \int q^2 \frac{n(\vec{r})n(\vec{r}')}{|\vec{r} - \vec{r}'|}d\vec{r}d\vec{r}' + E_{xc}[n] - \int V_{xc}(\vec{r})n(\vec{r})d\vec{r}.
\end{equation}
This is the fundamental working principle of the density functional theory and Kohn-Sham equations.

\section{Limitations of DFT}
\begin{itemize}
    \item Local minima method
    \item Not exact $V_{xc}$, means we must compromise between accuracy and cost, and choose between the different methods for specific application. There is no one best overall method that is superior for all purposes. 
    \item Not exact kohn-sham eigenfunctions, meaning band gap inaccurate
    \item Self-interaction term 
\end{itemize}

