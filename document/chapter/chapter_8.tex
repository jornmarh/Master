\chapter{The bad (Other composistions)}
\label{sec:bad}

In similar fashion to the preceeding section, we here list the mean and standard devitation of a set of SQSs for different high-entopy silicides of the fesi2 unit cell. The composistions we have selected in this segment is to a degree arbitratry aside from being 3d elements wi simply wish to observe the outcome of different alloys in respect to the band gap. The composistions are
are eqvimolar distributions of \ch{CrFeCoNiSi2} and \ch{CrFeTiNi} where manganese is replaced by Cobalt, then Titanium. Furthermore \ch{CrFeMnTiSi2} where the large Ni atoms are replaced by much smaller Ti, and \ch{CoFeMnNiSi2} where Cobalt takes the place of Chromium. Keep in mind, that these SQSs were not simply created by replacing elements in the existing supercells of CFMN (fesi2), but rather generated through TDEP and the SQS theory to form 5 new distinct supercells per composistion out of the fesi2 unit cell, identical to how we tested CFMN (fesi2). Additontly, all supercells contain 48 total atoms as before.  

\begin{table}[h!]
\centering
\begin{tabular}{@{}cccccc@{}}
\toprule
         & \multicolumn{2}{c}{Toten (eV)} & Enthalpy of formation &  \multicolumn{2}{c}{Mag} \\ \midrule
CrFeCoNi & - 6.4655        & 0.0056     & -12.7536       & 0.0083     & 0.0155     \\
CoFeMnNi & - 6.4731        & 0.0046     & -15.0836       & 0.0000    & 0.0000          \\
CrFeTiNi & - 6.4217        & 0.0087     & -7.5040       & 0.0305     & 0.0293     \\
CrFeMnTi & -6.6994         & 0.0071     & -7.3060       & 0.1142     & 0.0641     \\ \bottomrule
\end{tabular}
\end{table}

The first observation we make of table .. is that the total energy is much lower than CFMN, ie less stable \textbf{?}. The magnetization follows in line with the topics discussed above, clearly the magnetization is drasticly reduced by removing either manganese or chromium. \textbf{Vent på svar fra O.M om å sammenligne energier/entalpi på tvers av sammensetninger}   

Of the total 20 supercells we calculated in this compositonal spectrum, the overwelming majority are metals. In the CrFeCoNi composistion all SQSs were metalic, we perfomred additional calculations with the SCAN functional that corresponded well with the observed outcome from PBE. From the CrFeTiNi alloy we include the results of SQS A and B, where A display very small gaps in both spin channels with PBE of 0.039 eV and 0.002 eV, but no total band gap. This structure contain defect states in both channels, moreover the spin gaps are not observed in the density of states. However this can could be becouse of low resolution in the density of states, as we observed in structure C in the CFMN (fesi2) SQS and some of the permutations. Our last meassure of calculating the bad gap from the pymatgen package point to a metal. From these conflicted results we can not with certainty confirm the observed band gaps in this SQS, as we did in CFMN were all methods were in agreement.  We observe a smiliar case in B, the eigenvalues indicate a small total band gap of 0.008 eV and spin polarized gaps around 0.01 eV, in despite of defect states. But as in A, the gap is not found from other methods of evaluation or functionals. 

In the two previous examples we replaced manganese with Co and Ti and found very limited sucsess in regards to our task of finding semiconducting high-entropy silicides. Replacing Cr with Co instead, yielded two semiconducting SQSs A and E. We note that these supercells lie around the mean total energy of the set, but the upmost stable SQSs are metals. The band gap in A is in good agreement throughout all 3 meassures to a value of 0.033 eV with PBE and 0.041 with SCAN. What's more the gap transistion is consistent across both functionals in contrast to the results of  CFMN (fesi2). Surprisingly, the band gap vanishes with HSE06 calculations (TBC). Using gaussian smearing with smearing width 0.005 and 20000 points to calculate the DOS yields a band gap of 0.039 eV. We have found previosly for CFMN (fesi2) that these calculations typically are in good agreement with TBC and reliable. On the other hand, the gap is not entierly visable in the density of states, and not found from bandgap.py, but this could be a consequence of the inaccuracy in DOS calculations both with gaussian smearing and a small number of k-points.

In E, we find a band gap of 0.0058 eV from both the eigenvalues and bandgap.py with PBE, and a lesser gap of 0.0037 eV with SCAN. The density of states is not zero at Ef as seen in figure\textbf{insert figure}, but show very small values that may indicate a band gap as we ahave experienced in other cases as well. On the grounds that we find complemetary band gaps with both SCAN and PBE from eigenvalues and bandgap.py and no defect states in the eigenvalues, we conclude that this band gap is legitimate despite of the DOS values. We relate this result to the low resolution of 2500 points in calculating the density of states, considering that the gap is very narrow. \textbf{Wait for HSE06}

In the final composistion of fesi2 based high-entropy silicides tested in this project, we replace Ni atoms with much smaller Ti. As stated in the introduction we find the highest magnetic moment of all composistions with this arrangement, due to the inclusion of both chromum and manganese in the structure. \textbf{Most stable?} Among the 5 SQSs of this particlar mix, there is a significant variation in the magnetic moment between supercells, the lowest observed magnetic moment is 0.04 in E, and the highest 0.20 in B. We recognize a clear distinction between stability and magnetisim in this compound, the 3 lesser magnetic SQSs have very comperable hig total energy, and the two highly magnetic supercells are notebly less stable. 


\textbf{Write in the conclusion somewhere: In this project, we have read the band gap from our own written script where we evaluate the eigenvalues from the EIGENVAL file in vasp, used a script called bandgap.py from pymatgen, and considered the density of states. In the structures where we report a band gap, all these 3 have corresponded, additiontly the SCAN functional all though most frequently alter the value, agree on the presence of a band gap. However, we have found other structures where the eigenvalues show a small band gap. But due to the fact that the eigenvalues contain defect states, and nonphysical occupancy, the valididty of these resuslts are uncertain, and we do not include them unless they are observable from the other methods mentioned above. We will however include some of these in the discussion. }
