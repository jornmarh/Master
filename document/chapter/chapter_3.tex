\chapter{Special quasi-random Structures}
\label{sec:SQS}

The structure of high-entropy alloys in which the alloying elements occupy lattice sites by a random probability pose a problem on the numerical methods used for modeling. DFT in particular rely heavily on the periodicity in crystalline solids, as we will discover later. Some of the popular current tools for overcoming this problem of modeling a chemically disordered compound in the framework of density functional theory, are the virtual crystal approximation (VCA), Coherent Potential approximation (CPA), special quasi-random structure (SQS), and hybrid monte-carlo/molecular dynamics. (MC/MD). A brief overview of the different models is given in for example \cite{sqsIntro}. In this project we will exclusively use SQS to model random alloys, in large part from it's easy to use implementation and interpretation in VASP compared to the other options. However, SQS does offer certain benefits that will become clear throughout the following sections.             

\section{The fundamentals of SQS}
Before the arrival of SQS and CPA methods, the common approach of modeling random alloys was to distribute the numerous elements randomly over the lattice sites. This was a costly operation, which either involved averaging a great number of possible configurations, or infeasible large supercells considering the computational efforts required. In the original paper on SQS published in 1990 \cite{sqsfull}, it was proposed a selective occupation strategy to design special periodic quasi-random structures that exceed previous methods in accuracy and cost. The key concept was to create a periodic unit cell of the various components in a finite N lattice site single configuration such that the structure most closely resemble the configuration average of an infinite perfect random alloy. In an attempt to work withing the 50 lattice sites boundary of ab initio methods at that time. The working theory was that if one can resemble an infinite perfect random alloy by a periodic finite N cell, also the electronic properties would be similar between the two. The solution to this model was that for each N, ie lattice site, to minimize the difference of structural correlation function between the approximated cell and the perfect random alloy. There are obviously errors involved with approximating a random alloy by a periodic cell, but by the hierarchical relation to the properties of the material, interactions between distant sites only offer a negligible small contribution to the total energy of the system. Thus the aim of the SQS method is focused around optimizing the correlations within the first few shells of a given site. To follow is a review of the mathematical description of special quasi-random structures.

\section{Mathematical formulation}
We begin this section by giving a brief review of topics such as cluster expansions, statistics and superposition of periodic structures. A broader description of these topics can be found in the original article, or elsewhere in the literature. On a side note regarding the following mathematical derivation, the original concept was devolved in mind of an random binary alloy, but the theory have late successfully been extended to multi-component alloys and other special cases. 

The different possible atomic arrangements are denoted as "configurations" $\sigma$. The various physical properties of a given configuration is $E(\sigma)$, and $<E>$ is the ensemble average over all configurations $\sigma$. In practice, this quantity is unfeasible in terms of computational cost, seeing as the average require calculations and relaxations of all possible configurations, for a binary alloy this is $2^N$ for a fixed N number of lattice sites. A solution to this is to use the theory of cluster expansions and discretize each configuration into "figures" $f$. A figure in the lattice is defined in terms of the number of atoms it include $k$, distance in terms of neighbors $m$, and position in the lattice $l$. Further we assign spin values for each lattice site $i$ in the figure to denote which element it holds (+1,-1 for a binary alloy). By defining the spin product of spin variables in a figure at lattice position $l$ as $\Pi_f(l, \sigma)$, we can write the average of all locations in the lattice of a given figure $f$ as 
\begin{equation}
    \boldsymbol{\Pi}_f(\sigma) = \frac{1}{ND_f} \sum_l \Pi_f (l,\sigma)
\end{equation}
where $D_f$ is the number of equivalent figures $f$ per site. The brilliance of this notation is that we now can express the physical property $E(\sigma)$ in terms of the individual contributions $\epsilon_f$ of a figure f.
\begin{equation}
    E(\sigma) = \sum_{f,l} \Pi_f(l, \sigma) \epsilon_f(l)
\end{equation}
The quantity $\epsilon_f$ is called the "effective cluster property" and is defined as (for a random binary alloy $A_{1-x}B_x$)
\begin{equation}
    \epsilon_f(l) = 2^{-N}\sum_\sigma^{2^N} \Pi_f (l,\sigma) E(\sigma)
\end{equation}
Inserting the equation for $\boldsymbol{\Pi}_f$ into that of $E(\sigma)$ we can describe the the previous cluster expansion of $E(\sigma)$ as
\begin{equation}
    E = N\sum_f D_f<\boldsymbol{\Pi_f}>\epsilon_f
\end{equation}
And obtain a simplified expression for $<E(\sigma>)$ in eq 1? Thus we have successfully managed to reduce the expensive task of sampling all $E(\sigma)$ into calculating the effective cluster properties and summing over all types of figures. Remembering that $E(\sigma)$ can relate to many physical properties, the most common and applied case is that $E(\sigma)$ is the total energy, while $\epsilon_f$ is many body interaction energies. The cluster expansion above converge rather quickly with increasing number of figures, an effective method is thus to select a set of configurations to evaluate the effective cluster properties. Don't know how to write this, but the next step is to select a finite largest figure denoted $F$, and "specialize" the cluster expansion to a set of $N_s$ periodic structures ${\sigma} = ${s} to obtain the two expressions for $E(s)$ and $\epsilon_f$ using matrix inversion to obtain the result for $\epsilon_f$
\begin{align}
E(s) &= N\sum_{f}^{F} D_f \boldsymbol{\Pi}_f (s)\epsilon_f \\     
\epsilon_f &= \frac{1}{ND}\sum_{s}^{N_s}[\boldsymbol{\Pi}_f (s)]-1E(s)
\end{align}
Assuming now that the sum of figures $F$ and $N_s$ periodic structures are well converged, $E(\sigma)$ can be rewritten as a superposition of $E(s)$
\begin{align}
    E(\sigma) &= \sum_{s}^{N_s}\xi_s(\sigma)E(s) \\
    \xi_s(\sigma) &= \sum_{f}^{F}[\boldsymbol{\Pi}_f(s)]^{-1}\boldsymbol{\Pi}_f(\sigma)    
\end{align}
where $\xi$ is the weights. Thus we have effectively reduced the problem to a convergence problem of the number of figures $F$ and structures $N_s$. This can be easily solved given that we are dealing with periodic crystal structures ${s}$ that can employ the general applications of ordered structures from ab initio methods, and increasing $F$ until the truncation error falls bellow a desired threshold. However, this approach requires that the variance of the observable property is much lower than the sample mean, otherwise one would have to employ a much bigger sample size to reach statistical convergence. Don't how to write this part nicely, but: Because of the different relationship between various physical properties and the correlation functions, one observe different convergence depending on the meaning of $E$. The idea behind SQS was therefore to design single special structures with correlation functions ${\boldsymbol{\Pi}_f(s)}$ that most accurately match those of the ensemble average of a random alloy $<\boldsymbol{\Pi}_f>_R$. 

The correlation functions of an perfect random infinite alloy, denoted as $R$ is defined bellow
\begin{equation}
    \boldsymbol{\Pi}_{k,m}(R) = <\boldsymbol{\Pi}_{k,m}>_R = (2x-1)^l 
\end{equation}
with $k, m$ defined as before and x being the composition ratio of the alloy. In the case of an eqvimolar alloy ($x=\frac{1}{2}$), the functions equal 0 for all $k$ except $<\boldsymbol{\Pi}_{0,1}>_R = 1$. If we now randomly assign either atom A or B to every lattice site, for a sufficiently large value of N, the goal is then to create a single configuration that best match the random alloy. Keeping with the $x=\frac{1}{2}$ case, the problem is now that even though the average correlation functions of a large set of these structures approaches zero, like for the random alloy. The variance of the average is nonzero meaning that a selected structure of the sample is prone to contain errors. The extent of these errors can be evaluated from the standard deviations
\begin{equation}
    \nu_{k,m}(N) = |<\boldsymbol{\Pi}^{2}_{k,m}>|^{\frac{1}{2}} = (D_{k,m}N)^{-\frac{1}{2}}
\end{equation}
Given the computational aspects, it's obvious that economical structures with small N are prone to large errors. In fact, in some cases these errors can result in correlation functions centering around 1, as opposed to 0 for a perfect random alloy.  

I don't know how to write the prelude to this part! (see section IIIA in \cite{sqsfull}). The degree to which a structure $s$ fails to reproduce the property $E$ of the ensemble-averaged property of the random alloy can be described by a hierarchy of figures, see eq .. bellow
\begin{equation}
    <E> - E(s) = \sum_{k,m}' D_{k,m}[(2x-1)^k-\boldsymbol{\Pi}_{k,m}(s)]\epsilon_{k,m}
\end{equation}
, the prime is meant symbolize the absence of the value $0,1$ for $k,m$. The contribution from the figure property $\epsilon$ reduces for larger figures. In general, for disordered systems, the physical property "E" at a given point R falls of exponentially as $|R-R'|/L$, where L is a characteristic length scale relating to the specific property. Using this, the approach of SQS is to specify a set of correlation functions that hierarchically mimic the correlation functions of the random alloy. Meaning that it prioritize the nearest neighbor interactions. With the set of functions decided on, the objective it finally to locate the structures that correspond to the selected structures. 

With this approach, \cite{sqsfull} managed by mimicking the correlation functions exact for the first two shells, to reduce the computational measures of an accurate models. In this exact study they matched the results of an $N \rightarrow \infty$ by an $N=8$ SQS. In the final section of this chapter, we will take a look at the recent advances in the SQS method and application to high-entropy alloys. 

\section{Application of SQS to high-entropy alloys - Add figure}
The success of the SQS-approach is related to to the fact that we create simple periodic structures that allow for standard DFT based methods to calculate properties such as the total energy, charge density and electronic band structure \cite{sqs_dos}, \cite{sqs_bg}. However, the application to high-entropy alloys have been limited prior to the last couple of years on the grounds of computational demands involved in modeling disordered multi-component systems and the very recent emergence of the field in general. Mainly the last few years have we seen studies of high-entropy alloys based on special quasi-random structures \cite{WANG2021128754}, \cite{WEI2021167432}, \cite{RASHID2014285}, \cite{SORKIN2021160776}.   

From the pioneering work of  M.C Gao et al. \cite{hea2016_ch10}, whom in 2016 presented a comprehensive review high-entropy alloys modeling with SQS based on ab initio simulations in the framework of DFT and VASP,  majorly $CoCrFeNi$ and $CoCrFeMnNi$. It's apparent that the physical accuracy of SQS-models and their corresponding properties are very sensitive to the size of the supercell, or more generally the total number of atoms. Typically, cells with more than 64 atoms, often 125 or even 250 atoms were required to match experimental findings and results from other modeling techniques such as MC/MD and CPA. Examples are the predicted ground-state structure and pair distribution functions (PDFs). The key takeaway from the work of M.C Gao et al. can then be that in order for special quasi-random structures to accurately model the disordered structure of high-entropy alloys, a large number of atoms is required. Another point to this is that in addition to chemical disorder, HEAs possess a disordered magnetic structure. Considering the computational aspects, this means the optimal magnetic configuration is difficult to locate in high-entropy alloys and may contribute to the incorrect predictions in certain cases.

\textbf{How to I write this final paragraph of why SQS is used at HEAs today? mcSQS?}
Something on in very recent time, computational resources have increased and the generation of SQS have improved by the mc-SQS approach? This allows for a reduction in cell size and computation time, all though the convergence of SQSs and magnetic configuration remain troublesome in some cases. Maybe mention that the implementation of the TDEP package drastically improve the computational cost.     
