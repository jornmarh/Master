\chapter{High-Entropy alloys}
\label{sec:HEA}

To begin this project, we will start by giving a brief description of high-entropy alloys. This will include an introduction covering the basics and definitions, as well some more advanced topics relating to the physical properties of HEA's. This section will be largely based on the fantastic description of HEA's in "High-Entropy Alloys - Fundamentals and Application" \cite{HEA2016} and the references therein, it's an excellent read.

\section{Fundamentals}
High-Entropy Alloys are a quickly emerging field in materials science due to the infinitely many possibilities and the fascinatingly observed properties. Since it's original discovery by Jin in 2004, as of 2015 there have been over 1000 published journal articles on the field. 
In its simplicity, a high-entropy alloy can be compared to a smoothie. By combining an assortment of fresh fruit and vegetables one can produce unique combinations of flavors and nutritional values based on both the properties of the distinct items, and their interplay in the mixture. In materials science, this exact procedure can be applied to generate a large range of materials with tunable properties depending on the intended application. In the topic of HEA's, this can be increased strength or ductility, corrosive resistance or lowered thermal conductivity, all of which have been observed in actual researched high-entropy alloys. 
Moving on from the rather banal fruit analogy, a high-entropy alloy typically falls under the two conditions.
\begin{enumerate}
    \item The material consist of at least 5 distinct elements, where each element contribute between 5-35$\%$ of the composition
    \item The total configurational entropy is greater than 1.5R, where R is the gas constant. 
\end{enumerate}
The latter in particular is an especial case for high-entropy alloys. Typically the configurational entropy, eq .. for a solid solution increases with a higher number of components in the mix. For example, the ideal configurational entropy of binary alloy is 0.69R, while a 5-component alloy is 1.61R \textbf{ref book}. If we neglect other factors that influence the formation of solid solutions (will be covered later), from Gibbs free energy (eq ..) the two primary factors in impacting the formation of solid solution is the mixing enthalpy, which is the driving force to form compounds, and the mixing entropy which is the driving force to form random solid solutions. At elevated temperatures especially, it's clear that the energy term related to the entropy of the system becomes comparative to the mixing enthalpy and contribute to the overall equation. In summary, the overall concept of high-entropy alloys is that through alloying a greater number of elements, the increase in the configurational entropy of the system prohibit formation of intermetallic compounds in favor of a random solid solution. The random term simply relate to the various components occupying lattice positions based on probability. 
All though the mixing entropy as mentioned above plays a central role in the formation, there are many more factors that may oppose the formation of a single disordered phase. One of these is the atomic size effect which is related to the discrepancy in atomic sizes between the various elements in the alloy. \textbf{The book} points to a study, and figure illustrating the relationship between $\Delta H_\text{mix}$ and $\delta$, indicating that for very small values of $\delta$, ie similar atomic sizes. The elements have an equal probability to occupy lattice sites to form solid solutions, but the mixing enthalpy is not negative enough to promote formation of solid solution. Increasing $\delta$ does result in greater $\Delta H_\text{mix}$, but leads to a higher degree of ordering. In conclusion, the formation of solid solution high-entropy alloys lie in a narrow range of $\delta$ value to satisfy both the enthalpy of mixing and the disordered state. To define this range of formation of HEA, Yang and Zhang introduced the parameter $\Omega$ as
\begin{equation}
\Omega = \frac{T_\text{m} \delta S_\text{mix}}{|\Delta H_\text{mix}|}
\end{equation}
, and found that the formation of single disordered solid solution is found for $\Omega \geq 1.1$ and $\delta \leq 6.6\%$. While compounds such as intermetallics form for greater values of $\delta$ and lesser values of $\Omega$. Similarly, replacing the atomic size effect constant for the number of elements result in an equivalent condition.
Another factor that is strongly related to the crystal structure and physical properties of high-entropy alloys is the total number of electrons VEC (Valence electron concentration). VEC is an important parameter that is strongly related to the crystal structure of the material. For example, in $Co_3V$, originally a hexagonal structure. Can by increasing the VEC by alloying with Ni transform to a tetragonal structure, or a cubic structure by reducing the VEC with Fe. Altering the crystal structure had a massive impact on the properties of the material, seeing as hexagonal alloys tend to be brittle, while cubic alloys are more ductile. In fact, Mizutani \textbf{book, cite?} obtained a clear relationship between physical properties in 3d transition.  metals and the VEC. Derived from the work of Guo et al. on the phase stability of a $Al_xCrCuFeNi_2$ HEA, the VEC can be directly related to the crystal structure of high-entropy alloys. A lower VEC stabilize the BCC phase, while higher values stabilize FCC. In between is a mixture of the two. Specifically values greater than 8.0 stabilize FCC, and values bellow 6.87 favor BCC. However, these boundaries is not rigid when including elements outside of transition metals, exceptions has also been found for high-entropy alloys containing "Mn". All though a heavy majority of reported high-entropy alloys that form solid solutions are either FCC or BCC, recent studies have observed HEA's in orthorombic structure like $Ti_{35}Zr_{27.5}Hf_{27.5}Ta_5Nb_5$ and hcp structures, for example $CoFeNiTi$

\section{Core effects}
Next, we will summarize the discussion above into four core elements that distinctly describe high-entropy alloys and their implications on the functional properties. The first of these is the "high-entropy effect", related to the increased configurational entropy due to the amount of elements, that can inhibit the formation of strongly ordered structures. The high-entropy effect is most prominent in Gibbs energy of a system, where the balance between entropy and enthalpy is central and the effects are further emphasized at elevated temperatures, responsible for the high-temperature stability in high-entropy alloys.  \textbf{This needs work!}. Secondly is "severe lattice distortion effect", that originates from the fact that every element in a high-entropy structure is surrounded by non-homogeneous elements, thus leading to lattice strain and stress. The overall lattice distortion is additionally attributed to the differences in atomic size, bonding energies and crystal structure tendencies between the components. Therefore the total lattice distortion observed in HEA's are significantly greater than that of conventional alloys. This effect mostly affect the strength and conductivity of the material, such that a higher degree of distortion yields greater strength and greatly reduces the electronic and thermal conductivity due to increased electron and phonon scattering. An upside to this is that the scattering and following properties become less temperature dependent given that it originates from the lattice rather than thermal vibrations. 
The two remaining effects, "sluggish diffusion" and "cocktail effect" can be summarized swiftly. The first is a direct consequence of the multi-component layout of high-entropy alloys, that result in significantly slowed diffusion and phase transformation because of the number of different elements that is demanded in the process. The most notable relation from this effect is an increased creep resistance. Lastly we have the cocktail effect, which is identical to the smoothie analogy mentioned previously, in that the resultant characteristics of such a mixture is a combination of both the elements and their interaction. This is possible the most promising concept behind high-entropy alloys, which fuels researchers with ambition to discover highly optimized materials by meticulously combining and predicting properties from different elements. Examples of this can be the refractory HEA's developed by "Air Force Research Laboratory" severely exceeding the melting points and strength of previous Ni or Co-based superalloys by alloying specifically refractory elements such as Mo. Nb and W. Or the research conducted by Zhang et al. on the high-entropy system $FeCoNi(AlSi_{0-0.8}$ in the intent of unveiling the optimal combination of magnetic, electric and mechanical properties. Resulting in an excellent soft magnet. 

\section{Physical properties}
In the discussion above of the core effects regarding high-entropy alloys, especially the lattice distortion and cocktail effect are responsible for the physical properties because of the former's effect on scattering and the latter on combining properties from various elements. The initial study on the physical properties on high-entropy alloys was conducted on the H-x alloy, referring to the system $Al_xCoCrFeNi$ with $0 \leq x \leq 2$. It was found that the electrical resistivity was higher than that of conventional alloys, and that the conductivity generally decreased with increasing amounts of Al. Going back to the section on lattice distortion, we also know that these properties are relatively temperature insensitivity since the lattice effects outweigh the thermal effects in terms of electron and phonon scattering. To further justify this point, the H-x compound mentioned above exhibited noteworthy low carrier mobility compared to conventional alloys. Similar findings have also been made for the $FeCoNi(AlSi)_x$ system.
\textbf{Write a small part on magnetic and relate to the cocktail effect, then very briefly conclude by mentioning findings of superconductivity, corrosion resistance, hydrogen storage and other properties/applications.}