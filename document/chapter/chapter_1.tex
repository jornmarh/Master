\chapter{High-Entropy Silicides}
\label{sec:lab}

The structure of this project will begin by a review of high entropy alloys and silicides, intended to cover the relevant theory behind these material classes relevant for this project. In particular, we will look at the binary silicides CrSi2, FeSi2, and Fe2Si, amidst other relevant examples. Thereafter we intend to dive into the world of physics, covering topics such as quantum mechanics, solid state physics and density functional theory, with primary emphasis on energy bands and the band gap of solids. Before transitioning into the methods used in this study such as special quasirandom structures, meta-GGA and hybrid functionals

\section{High-Entropy Alloys}
We initialize the theoretic portion of this article with a review of high-entropy alloys (HEA). The following section is largely based on the description in High-Entropy Alloys: Fundamentals and Application [1] and references therein.
High-entropy alloys are a quickly emerging class of materials with both mechanical and functional applications due to interesting properties. The first publications on HEA’s were published in 2004 by Professors Jien-Wei Yeh and Brian Cantor. Since then, high-entropy alloys have been an increasingly popular field of research within material science for the last two decades. Point to research growth on HEA’s last two decades. A good point here is as said in the book, in the evolution of material science from stone and wood to the technology of today due to alloying. Additionally, the food analogy to alloying is a good point. I am not sure to put this here or in the introduction. Maybe both?
The concept of high-entropy alloys is rather simple, a high-entropy alloy is simply an extension of traditional binary alloys such as steel (Iron + steel). But the underlying physics and precise definition demands a bit more complexity. Originating from its name, a central concept of HEA’s Is the configurational entropy of the system defined by Boltzmann’s equation


\section{Silicides}
i am citing this \cite[95]{AM69}

