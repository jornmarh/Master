\textbf{Change this introduction to fit the final product!}
In this one year long project, we have collected results of a great number of materials with various structures and compositions. The initial experimentation was based on high-entropy silicides of the $Fe_2Si$ unit cell, created from the special quasi-random structure approach as described above. Despite the non-semiconducting character of this compound, we worked under the idea that the extraordinary properties that have been observed in high-entropy alloys through effects such as the cocktail effect, we could discover specific combinations of elements that would yield a semiconductor. In addition, the ratio between silicon atoms to metals allowed us to create nearly eqvimolar high-entropy alloys. 

Following this attempt, we transitioned into studying high-entropy silicides based on well known semiconducting 3d silicides such as $\beta-$\ch{FeSi2}, \ch{CrSi2} and \ch{MnSi_{1.75}}. The main outcome of this project is that for all 4 different starting silicides, we could only produce high-entropy silicides from one unit cell, furthermore in this cell only one specific compositions of elements was semiconducting. This was \ch{Cr_{0.25}Fe{0.25}Mn_{0.25}Ni_{0.25})Si2}, here-in CFMN, in the $\beta-$ \ch{FeSi2} crystal structure.  

This section will be structured in the following manner, firstly we will investigate the CFMN (fesi2) compound and various permutations of the composition. Thereafter we will look at other possible compositions of fesi2 based high-entropy silicides, and lastly test the CFMN composition in other crystal symmetries. In final we will present an overview of the complete study and the various compounds that have been investigated in order to propose promising directions and guideline future research directions in this field. In this way, we aim to understand the uniqe properties of CFMN (fesi2) and why this particular compound is semiconducting compared to the other testes structures in this project. Properties we will cover is the overall stability by total energy and corresponding enthalpy of formation, the magnetic properties and which elements contribute to the magnetism. But in majority, we will look at the band gap and related properties, as this is the main motivation and distinction of the study.


\chapter{\ch{(CrFeMnNi)Si2} in the $\beta-$\ch{FeSi2} structure}
\label{sec:equi}

\section{Bulk $\beta-$ \ch{FeSi2}}
$\beta-FeSi_2$ in the orthorombic cmce crystal lattice is a well known semiconductor with an experimentally measured band gap of around 0.85 ev at room temperature \cite{PhysRevB.58.10389}. The nature of the band gap is under debate, all though most ab inito studies point to an indirect gap, experimental work result in a direct gap. In our study we find an indirect band gap of 0.65 eV with PBE, in comparison materials project find a band gap of 0.70 eV. This slight discrepancy is most likely down to use of different parameters in the calculations, for example the cutoff energy or number of k-points. In accordance with materials project we find that this compound is nonmagnetic. This can be seen from the electronic density of states in figure 7.1 by that the DOS and hence band gap is identical in both spins.   

\begin{figure}[H]
\centering
\includegraphics[width=\textwidth]{results/fesi2/bulk_DOS.png}
\caption{Density of states (PBE) $\beta-$ \ch{FeSi2}}
\end{figure}

We calculate the enthalpy of formation $\Delta H_f$ as the difference in total energy between the product and sum of reactants. For the FeSi2 compound consisting of 16 iron atom sand 32 silicon we get $\Delta H_f = -327.72 eV - (16 \cdot -8.32 eV + 32 \cdot - -5.42 eV = -21.16 eV$, where the total energy per atom of iron and silicon was calculated separately for the respective base elements with identical parameters as used for the \ch{FeSi2} calculation. Typically $\Delta H_f$ is listed in $kJmol^{-1}$, thus we get in final an enhalpy of formation $Delta H_f = - 42.552 KjJmol^{-1}$. We note that this value is quite far away from other experimental results, such as $-74 KJmol^{-1}$ in \cite{entalpi}. However this value correspond to the material at room temperature, while we have only looked at the ground state. 

\section{\ch{(CrFeMnNi)Si2} SQSs}
Details on both alloys and the bulk \ch{FeSi2} structure is covered in section 6.2, the SQSs can be seen in figure 6.1. Below in table 7.1 we list the total energy per atom (Toten), final magnetic moment (Mag) and band gap ($E_G$) of the five distinct SQSs, plus the corresponding enthalpy of formation $\Delta H$ calculated from the mean total energy. For simplicity we denote the 5 supercells as A, B, C, D and E. 

\begin{table}[H]
\centering
\begin{tabular}{@{}cccc@{}}
\toprule
SQS           & \begin{tabular}[c]{@{}c@{}}Toten \\ (eV)\end{tabular} & \begin{tabular}[c]{@{}c@{}}Mag \\ ($\mu_B$)\end{tabular} & \begin{tabular}[c]{@{}c@{}}$E_G$ \\ (eV)\end{tabular} \\ \midrule
A             & −6,6080                                               & 0.0833                                                   & 0.0280                                                \\
B             & −6,6138                                               & 0.0833                                                   & 0.0523                                                \\
C             & −6,6063                                               & 0.0834                                                   & 0.0344                                                \\
D             & −6,6155                                               & 0.0833                                                   & 0                                                     \\
E             & −6,6089                                               & 0.0833                                                   & 0.0495                                                \\ \midrule
Mean          & -6.6105                                               & 0.0833                                                   & 0.0328                                                \\
Std           & 0.0039                                                & 0.0000                                                   & 0.0210                                                \\
$\Delta H$   & -11.5000 eV                                           & -                                                        & -                                                     \\ \bottomrule
\end{tabular}
\caption{Total energy per atom, final magnetic moment and band gap of 5 unique SQS of \ch{(CrFeMnNi)Si2} based on the $\beta-$ \ch{FeSi2} unit cell.}
\end{table}

In addition to the mean energy we include the standard deviation "std" of the set to observe how far each supercell is from the mean value for different properties. Clearly both the total energy and magnetic moment show very little variation if any between SQSs which is expected seeing as the only difference is the atomic configuration. Contrary we observe greater variation in the band gap between SQSs, ranging from 0.02 - 0.05 eV, nevertheless much smaller than the the parent bulk material. On the other side we find that the magnetic moment has increased compared to in the bulk structure. Seen throughout all SQSs, the local magnetic moments is the largest for Cr followed by Mn, on the other hand the ferromagnetic elements Fe and Ni have small negative moments. This is an odd result, in calculations with identical parameters of the base elements we find that Cr is non, Fe and ni is .., same with manganese. In HEAs specifically we saw in section 2.2 that in several cases particularly chromium reduce the magnetization of the compound. There are however several uncertainties concerning the listed magnetic values in this project, and the reported values are meant to be superficial indications of the magnetism in the real material. For example we have the limitations mentioned previously about both DFT and special quasi-random structures to model magnetic and particularly paramagnetic materials. The latter especially is as an important factor, as the outcome of the base elements are in much better agreement with other experimental data, such as materials project. This could be related to the SQSs distinct for this project, where we kept the Si sites in the lattice constant and restricted the 3d elements to occupy exclusively the Fe sites in the lattice. Such restrictions could impact the magnetic interactions between elements compared to in a real random alloy.  Lastly each SQS have been tested solely with co-linear spin polarization, thus negleticng the many other possible configurations. The precise magnetic properties was compromised due to the large computational demand of locating the optimal magnetic configuration of each SQS and composition, and the focus of this project leaning more towards the band gap of the structures.   

  
In terms of the total energy the most and least stable SQSs are "D" and "A" respectively. Based on the total energy between SQSs, D is then the most representative configuration of the real material. However most likely all five SQSs and other possible SQSs would appear as local orderings in domains of the real material with a certain probability. Therfore we will consider the results of the other SQSs as well as the most stable supercell. Furthermore the total energy alone is not sufficient to evaluate the stability. In this project we have not considered factors such as the confirational entropy or the optimized magnetic configuration of each structure. Additionally we have only studied the ground state and thus have no knowledge of the properties at finite temperatures. Hence the relationships and properties between the five SQSs listed in table 7.1 are not rigid.     

\newpage
\subsection{The band gap}
As seen in table 7.1, the band gap of the alloys are severely reduced from the bulk material, and show variation between each of the five SQSs. We observed a maximum band gap of 0.05 eV in SQS B, and on the flip side a 0 band gap in SQS D. The density of states of the respective structures can be seen below in fiugres 7.2 and 7.3.  

\begin{figure}[H]
	\centering
	\includegraphics[width=\textwidth]{results/fesi2/D_TDOS.png}
	\caption{Density of states of SQS D \ch{(CrFeMnNi)Si2} with PBE.}
\end{figure}

\begin{figure}[H]
\centering
	\includegraphics[width=\textwidth]{results/fesi2/B_TDOS.png}
	\caption{Density of states of SQS B \ch{(CrFeMnNi)Si2} with PBE.}
\end{figure}  

In figure 7.2  and 7.3 we observe that the band gap in both SQS D and B in accordance with the magnetic property is different between spins. Going forward we will refer to the band gap in spin up as $E_G ^\text{up}$, and spin down as $E_G ^\text{dw}$. Clearly in both D and B $E_G ^\text{up} > E_G ^\text{dw}$ with no gap in spin down in D and 0.05 eV in B compared to around 0.3 eV in spin up in both structures. Comparing to the values listed in table 7.1 it's clear that the total band gaps of the respective structures correspond to the narrow or nonexistent band gap in spin down. To obtain further and more precise information on the band gap we look to the calculated Kohn-Sham eigenvalues. The eigenvalue band gaps, denoted as $E_G ^\text{eigen}$ can be seen below in table 7.2 for all five SQS, the density of states of SQS A, C and E can be found in appendix [REF].  

\begin{table}[H]
\centering
\begin{tabular}{@{}cccc@{}}
\toprule
SQS & \begin{tabular}[c]{@{}c@{}}$E_G ^\text{up, eigen}$ \\ (meV)\end{tabular} & \begin{tabular}[c]{@{}c@{}}$E_G ^\text{dw, eigen}$ \\ (meV)\end{tabular} & \begin{tabular}[c]{@{}c@{}}$E_G ^\text{tot, eigen}$ \\ (meV)\end{tabular} \\ \midrule
A   & 81.4                                                                     & 52.2                                                                     & 28.1                                                                    \\
B   & 293                                                                      & 52.2                                                                     & 52.2                                                                    \\
C   & 236                                                                      & 34.3                                                                     & 34.3                                                                    \\
\textbf{D}   & 339                                                                      & 0.00                                                                     & 0.00                                                                    \\
E   & 308                                                                      & 50.0                                                                     & 50.0                                                                    \\ \bottomrule
\end{tabular}
\caption{Band gap of the 5 SQSs of \ch{(CrFeMnNi)Si2} calculated from the eigenvalues in spin up, down and total.}
\end{table}

We find a continuing trend across all five SQS similar to D and B where the band gap is limited by the spin down channel, as all SQSs par A display much greater values in spin up. In VASP the energy eigenvalues are listed for every energy band at all k-points used in the calculation, with corresponding occupancy. An occupancy of 1 represents a fully occupied eigenstate, in analog an empty eigenstate have occupancy equal to 0. Recalling that occupied states belong to the valence band, and the conduction band consists of unoccupied states. The highest energy valence band in these SQSs is band 124 in spin down and 128 in spin up, following the lowest energy valence band is 125/129 in spin down/up. The band gap in spin down is then determined from the difference between the lowest energy eigenvalue in band 125 and the highest energy eigenvalue in band 124, and likewise for the spin up band gap between bands 128 and 129. In SQS D different from the semiconducting structures we observe some partially occupied states at the band edges in bands 124 and 125 in spin down. With partially occupied states we refer to eigenstates in the valence band with occupancy less than 1 and states in the conduction band with occupancy above 0. Specifically the highest energy eigenvalue (9.01 eV) in band 124 have occupancy equal to 0.94, and equivalently the lowest eigenvalue (8.98 eV) in band 125 have occupancy equal to 0.08 in this structure (spin down). As seen from the respective energies this results in a 0 (negative) band gap. To highlight this further we consider the hypothetical band of the structure by introducing an occupancy cutoff parameter $occ$ to calculate the band gap.

If we posed a restriction on the eigenvalues and occupancy so that    

         


Alternatively we can investigate the band gap to greater depths from the eigenvalues. The highest energy occupied energy band in all SQSs is 124 in spin down and 128 in spin up, meaning that energy bands above are unoccupied. In the semiconducting SQS there is a direct transition between the highest energy conduction band to the lowest energy valence band, so that all eigenstates in band 124 and 128 are fully occupied, and likewise all states in band 125 and 129 are completely empty, hence we have a band gap in both spins.


above 0.00 and bellow. The case of occupancy either above or bellow completely full or completely empty is a nonphysical oddetity that we have not been able to narrow down to any specific factor, but have found indication that this could be a numerical error attributed to the use of the tetrahedon smearing method \cite{TBC_fermi} in metallic systems.   



 a numerical inaccuracy well-known to calculations that apply the Tetrahedron method with Bloch corrections and have no real impact on the results. The second case where we have partially filled eigenstates in the conduction band and not completely filled states valence band is a familiar term in random alloys \cite{PhysRevLett.104.236403} in which the forbidden energy gap is contaminated by defect states. To further study this effect we introduce $E_G ^\text{eigen}(occ)$ to represent the band gap calculated at a occupancy cutoff $occ$ in the eigenvalues. Such that $E_G ^\text{up, eigen}(0.99)$ is the band gap in spin calculated from the eigenvalues with occupancy above 0.99. Equivalently $E_G ^\text{dw, eigen}(0.01)$ is the spin down band gap from eigenvalues with occupancy below 0.01. Hence eigenvalues with occupancy above/below this criteria, which we denote as defect states, are not considered. Following we will list the occupancy parameter as a single value, such that $occ = 0.1$ represent occupancy equal to 1- 0.1 and 0 + 0.1. Applying this to SQS D we get the results listed below in table 7.3, note that $E_G ^\text{up, eigen}$ is constant from the fact that only the spin down states contain defects.

\begin{table}[H]
\centering
\begin{tabular}{@{}cccc@{}}
\toprule
occ              & \begin{tabular}[c]{@{}c@{}}$E_G ^\text{up, eigen}$ \\ (meV)\end{tabular} & \begin{tabular}[c]{@{}c@{}}$E_G ^\text{dw, eigen}$ \\ (meV)\end{tabular} & \begin{tabular}[c]{@{}c@{}}$E_G ^\text{tot, eigen}$ \\ (meV)\end{tabular} \\ \midrule
0.5              & 339                                                                      & 0                                                                        & 0                                                                       \\
0.05             & 339                                                                      & 21.0                                                                     & 21.0                                                                    \\
0.01             & 339                                                                      & 49.6                                                                     & 49.6                                                                    \\
0.001            & 339                                                                      & 73.3                                                                     & 73.3                                                                    \\
\textless 0.0001 & 339                                                                      & 85.7                                                                     & 85.7                                                                    \\ \bottomrule
\end{tabular}
\caption{Band gap of SQS D as a function of occupancy in the eigenvalues.}
\end{table}

Here we find that by introducing a stronger criteria on the eigenvalues, that the band gap firstly become finite at $occ = 0.05$ and converge to 0.085 eV by "removing" all defect states. This happens because if we neglect certain partially filled states in the conduction band, the lowest energy eigenstate in this band increase in energy and thus minimum distance between the conduction band and valence band increase, and likewise in the valence band. Comparing to the density of states band gap plotted in figure 7.1, it's apparent that $E_\text{G} ^\text{dos}$ correspond to $E_\text{G} ^\text{eigen}(0.5)$. 


It would have been instructive to visualize and analyze the energy bands by plotting the band structure. Unfortunately this is neither simple to perform or interpret in large supercells consisting of several elements and a large number of energy bands. One solution to this is to perform band-unfolding, but this did not work in conjunction with the TDEP implementation of the special quasi-random structures method.


\newpage
\subsection{Local and projected density of states}
  
\begin{figure}[H]
	\centering
	\includegraphics[width=.7\textwidth]{results/fesi2/D_LDOS_Si.png}
	\caption{Local density of states of Si (SQS D)}
\end{figure} 

\begin{figure}[H]
	\centering
	\includegraphics[width=\textwidth]{results/fesi2/D_LDOS.jpeg}
	\caption{Local density of states of (a) Cr, (b) Mn, (c) Fe, (d) Ni in SQS D.}
\end{figure}   
  
In the local density of states plotted in figure 7.3 we see that the s-electrons in Si occupy states in the lower energy regions and p electrons at slightly elevated energies closer to the Fermi energy, above $E_F$ states are occupied by both s and p electrons almost equally. Further, the local density of states of the transition metals chromium, manganese, iron and nickel in SQS D is displayed bellow in figure 7.4. In spin down, manganese is most dominant especially above $E_F$, but also bellow $E_F$. Likewise chromium show a strong presence above the Fermi energy in spin down. Both iron an Nicel show largest contribution at energies further from the Fermi energy, most notably bellow $E_F$. In the spin up channel we see a similar trend where chromium lie closest to $E_F$ followed by manganese then iron and lastly nickel at the lowest energies. Another interesting observation is that the the LDOS of iron and nickel is much more symmetric between spins, than Cr and Mn. Comparing to the LDOS of iron and silicon in bulk $\beta-$ \ch{FeSi2} \cite{doi:10.1063/1.346415} we find good agreement for both Fe and Si in this alloy .

\begin{figure}[H]
	\centering
	\includegraphics[width=.8\textwidth]{results/fesi2/D_PDOS.png}
	\caption{Projected density of states SQS D CFMN (fesi2) from PBE calculation}
\end{figure} 

Moreover the relative positions and interplay between 3d elements and silicon as shown in the projected density of states (figure 7.5) is in good agreement with observed trends in simpler Si-rich transition metal silicides \cite{lange1997electronic}. The electronic structure tends to be dominated by TM d electrons, and the valence band density of states are filled by non-bonding d states near $E_F$. The p-d hybridization between Si and TM elements typically fall about 6 eV bellow $E_F$ and Si $s$ states about 10 eV bellow. In our case we find that the Si states are pushed up closer to the fermi energy by random alloying of various 3d elements.    

  
\begin{figure}[H]
	\centering
	\begin{subfigure}{.45\textwidth}
			\includegraphics[width=\textwidth]{results/fesi2/D_PDOS_Ef.png}
			\caption{SQS D}		
	\end{subfigure}
	\hspace{0.5cm}
	\begin{subfigure}{.45\textwidth}
		\includegraphics[width=\textwidth]{results/fesi2/B_PDOS_Ef.png}
		\caption{SQS B}		
	\end{subfigure}
	\caption{Projected density of states of SQS D and B around $E_F$}
\end{figure}

Above we have included the PDOS of SQS D and B but focused around $E_F$, from these figures we find that the spin down channel in D contain a more dominant presence of manganese especially, and some chromium as compared to the semiconducting SQS B.  

\subsection{The band gap with SCAN and HSE06}
As expressed previously in this work we invoke 3 level of depths GGA (PBE), meta-GGA (SCAN) and hybrid functional (HSE06) to determine the band gap of the SQSs, these results are showcased in table 7.4. Note that we do not specify eigen/dos or the occupancy here, because par SQS D the eigenvalues does not contain defect states, hence $E_\text{G} ^\text{eigen}(0.5) = E_\text{G} ^\text{dos}$.

\begin{table}[H]
\centering
\begin{tabular}{@{}ccccc@{}}
\toprule
\multicolumn{1}{l}{SQS}                 & XC-functional & \begin{tabular}[c]{@{}c@{}}$E_G ^\text{up}$ \\ (eV)\end{tabular} & \begin{tabular}[c]{@{}c@{}}$E_G ^\text{dw}$ \\ (eV)\end{tabular} & \begin{tabular}[c]{@{}c@{}}$E_G ^\text{tot}$ \\ (eV)\end{tabular} \\ \midrule
\multicolumn{1}{c|}{\multirow{3}{*}{A}} & PBE           & 0.0815                                                           & 0.0521                                                           & 0.0281                                                            \\
\multicolumn{1}{c|}{}                   & SCAN          & 0                                                                & 0                                                                & 0                                                                 \\
\multicolumn{1}{c|}{}                   & HSE06         & 0.7084                                                           & 0.0261                                                           & 0.0261                                                            \\ \midrule
\multicolumn{1}{c|}{\multirow{3}{*}{B}} & PBE           & 0.2932                                                           & 0.0523                                                           & 0.0523                                                            \\
\multicolumn{1}{c|}{}                   & SCAN          & 0.1470                                                           & 0.0890                                                           & 0.0890                                                            \\
\multicolumn{1}{c|}{}                   & HSE06         & 0.2855                                                           & 0.1819                                                           & 0.1819                                                            \\ \midrule
\multicolumn{1}{c|}{\multirow{3}{*}{C}} & PBE           & 0.2355                                                           & 0.0343                                                           & 0.0343                                                            \\
\multicolumn{1}{c|}{}                   & SCAN          & 0.0690                                                           & 0.1124                                                           & 0.1124                                                            \\
\multicolumn{1}{c|}{}                   & HSE06         & 0.1744                                                           & 0.0328                                                           & 0.0196                                                            \\ \midrule
\multicolumn{1}{c|}{\multirow{3}{*}{D}} & PBE           & 0.3386                                                           & 0                                                                & 0                                                                 \\
\multicolumn{1}{c|}{}                   & SCAN          & 0                                                                & 0.1086                                                           & 0                                                                 \\
\multicolumn{1}{c|}{}                   & HSE06         & 0.3780                                                           & 0                                                                & 0                                                                 \\ \midrule
\multicolumn{1}{c|}{\multirow{3}{*}{E}} & PBE           & 0.3078                                                           & 0.0495                                                           & 0.0495                                                            \\
\multicolumn{1}{c|}{}                   & SCAN          & 0.1540                                                           & 0.1112                                                           & 0.1048                                                            \\
\multicolumn{1}{c|}{}                   & HSE06         & 0.5476                                                           & 0.0133                                                           & 0.0133                                                            \\ \bottomrule
\end{tabular}
\caption{Band gap calculated with PBE, SCAN and HSE06 XC-functionals of \ch{(CrFeMnNi)Si2} SQSs.}
\end{table}


We will begin dissecting table 7.4 by comparing SCAN to PBE. The first distinction we make notice of is in SQS A. In this supercell calculations with the SCAN functional predicts a metalic compound, contrary to the the PBE band gap of 0.03 eV. Alike the band gap of SQS D discussed previously, the 0 band gap in this structure with SCAN is caused by defect states. Neglecting such states and evaluating the band gap from just completely filled and empty eigenstates yield $E_\text{G, SCAN} ^{up, eigen}(0.99, 0.01) = 0.0316$ eV and $E_\text{G, SCAN} ^{dw, eigen}(0.99, 0.01) = 0.0531$ eV, and a resulting semiconductor with a band gao of 0.0316 eV. This value seems to agree better with the PBE band gap of this supercell, but we observe that $E_G ^\text{up}$ is larger in PBE.  This is a recurrent patter with SCAN across all five SQSs, where $E_\text{G, SCAN} ^\text{up} < E_\text{G, PBE} ^\text{up}$, and moreover $E_\text{G, SCAN} ^\text{dw} > E_\text{G, PBE} ^\text{dw}$. This can be seen in figure 7.7, where we plot the density of states of SQS E (a, b) and C (c, d). Note that the SCAN band gap in C have the opposite spin polarization compared to PBE, this is also the case in SQS D.

\begin{figure}[H]
	\begin{subfigure}{.5\textwidth}
		\includegraphics[width=\textwidth]{results/fesi2/E_DOS_pbe.png}
		\caption{SQS E PBE}
	\end{subfigure}
	\begin{subfigure}{.5\textwidth}
		\includegraphics[width=\textwidth]{results/fesi2/E_DOS_scan.png}
		\caption{SQS E SCAN}
	\end{subfigure}
	\begin{subfigure}{.5\textwidth}
		\includegraphics[width=\textwidth]{results/fesi2/C_DOS_pbe.png}
		\caption{SQS C PBE}
	\end{subfigure}
	\begin{subfigure}{.5\textwidth}
		\includegraphics[width=\textwidth]{results/fesi2/C_DOS_scan.png}
		\caption{SQS C SCAN}
	\end{subfigure}
	\caption{Density of states illustrating the band gaps from PBE and SCAN calculations for SQS E and D.}
\end{figure}

\begin{figure}[H]
	\centering	
	\includegraphics[width=\textwidth]{results/fesi2/B_TDOS_hse06.png}
	\caption{Density of states of SQS B with HSE06}
\end{figure}

With the HSE06 functional we observe the opposite trend in SQS A and E, where $E_\text{G, HSE06} ^\text{up} > E_\text{G, PBE} ^\text{up}$ and $E_\text{G, HSE06} ^\text{dw} < E_\text{G, PBE} ^\text{dw}$. But in other cases $E_\text{G, HSE06} ^\text{up}$ is lesser (SQS C) or similar to PBE (SQS B and D). On the other hand $E_\text{G, HSE06} ^\text{dw}$ is consistently smaller in all structures compared to PBE, with the exception of SQS B. In this structure the HSE06 functional predicts large band gaps in both spins, as seen from the density of states plotted in figure 7.8.    
 
As we covered in section 5.1, the hybrid functional is much more computationally demanding compared to SCAN and PBE. To reduce the cost of the HSE06 functional we performed such calculations in this project with a lower density of k-points, see section 6.1. The narrow mesh of k-points is an important factor to mention in relation to the HSE06 band gaps, that could lead to artificially exaggerated band gaps as the low density of k-points could fail to encapsulate the exact minimum transition between the valence band and conduction band.

\begin{table}[H]
\centering
\begin{tabular}{@{}lc@{}}
\toprule
XC-functional & \begin{tabular}[c]{@{}c@{}}Transition \\ (k-point)\end{tabular} \\ \midrule
PBE           & (0.250,0.000,0.250) $\rightarrow$ (0.000,0.000,0.000)           \\
SCAN          & (0.250,0.000,0.250) $\rightarrow$ (0.000,0.333,0.000)           \\
HSE06         & (0.500,0.000,0.000) $\rightarrow$ (0.000,0.000,0.000)           \\ \bottomrule
\end{tabular}
\caption{Minimum gap between k-point in valence band and conduction band in SQS B from PBE, SCAN and HSE06}
\end{table}

From table 7.5, we observe that all 3 functionals find different band gaps, a concerning factor is that the highest energy k-point in the valence band from PBE calculations (0.250, 0.000, 0.250) is not considered in the HSE06 calculation with the narrow grid of 2x2x2 k-points. Thus one may suspect that the HSE06 calculation overlook the minimum transition and hence return an enlarged band gap. This could for instance be the case in $E_\text{G, A} ^{up}$ and $E_\text{G, B} ^{dw}$ where HSE06 predicts much larger values compared to PBE. However without an experimental baseline of the structure, we can not conclude that this is the case. As in the other SQSs we find examples where HSE06 produce similar or lower values than PBE despite of the smaller number of k-points. The concept of sufficient k-points is especially an important matter in metallic systems, as these are known to require a more dense mesh of points to accurately map the Fermi-surface \cite{smear}. However, further investigation of the importance of k-points was not possible within the time-frame of this project.   

As stated in section 6.2, we did not manage to converge hybrid calculations with the tetrahedron method, and overcame this problem by first calculating the charge density with Gaussian smearing and utilize the density to expedite calculation with TBC. The respective band gaps from these methods are displayed in table 7.6 for the five SQSs of the \ch{(CrFeMnNi)Si2} system. Here we calculate the band gap from the eigenvalues at different cutoff occupancy $occ$ to highlight the part of defect states. Calculations with Gaussian smearing was tested with smearing width $sigma$ equal to 0.05 eV and 0.005 eV.

\newpage
\begin{landscape}
\begin{table}[]
\vskip-1.5cm \hskip1cm \begin{tabular}{@{}cccccccc@{}}
\toprule
\multicolumn{1}{l}{SQS}                          & \begin{tabular}[c]{@{}c@{}}Smearing (type) \\ width (eV) \end{tabular} & \begin{tabular}[c]{@{}c@{}}$E_\text{G} ^{up, eigen}(0.5)$\\ (eV)\end{tabular} & \begin{tabular}[c]{@{}c@{}}$E_\text{G} ^{dw, eigen}(0.5)$\\ (eV)\end{tabular} & \begin{tabular}[c]{@{}c@{}}$E_\text{G} ^{up, eigen}(0.99)$\\ (eV)\end{tabular} & \begin{tabular}[c]{@{}c@{}}$E_\text{G} ^{dw, eigen}(0.01)$\\ (eV)\end{tabular} & \begin{tabular}[c]{@{}c@{}}$E_\text{G} ^{tot, eigen}(0.5)$\\ (eV)\end{tabular} & \begin{tabular}[c]{@{}c@{}}$E_\text{G} ^{tot, eigen}(0.99, 0.01)$\\ (eV)\end{tabular} \\ \midrule
\multicolumn{1}{c|}{\multirow{3}{*}{A}}          & \begin{tabular}[c]{@{}c@{}}Gaussian \\ (0.05)\end{tabular}    & 0.7837                                                                        & 0.1493                                                                        & -                                                                              & 0.2984                                                                         & 0.1493                                                                         & 0.2984                                                                                \\
\multicolumn{1}{c|}{}                            & \begin{tabular}[c]{@{}c@{}}Gaussian \\ (0.005)\end{tabular}   & 0.2117                                                                        & 0.1013                                                                        & -                                                                              & -                                                                              & 0.1013                                                                         & -                                                                                     \\
\multicolumn{1}{c|}{}                            & TBC                                                           & 0.7084                                                                        & 0.0261                                                                        & -                                                                              & -                                                                              & 0.0261                                                                         & -                                                                                     \\ \midrule
\multicolumn{1}{c|}{\multirow{3}{*}{B}}          & \begin{tabular}[c]{@{}c@{}}Gaussian \\ (0.05)\end{tabular}    & 0.2783                                                                        & 0.1702                                                                        & 0.2988                                                                         & 0.3136                                                                         & 0.1506                                                                         & 0.2979                                                                                \\
\multicolumn{1}{c|}{}                            & \begin{tabular}[c]{@{}c@{}}Gaussian \\ (0.005)\end{tabular}   & 0.2838                                                                        & 0.1823                                                                        & -                                                                              & -                                                                              & 0.1801                                                                         & -                                                                                     \\
\multicolumn{1}{c|}{}                            & TBC                                                           & 0.2855                                                                        & 0.1819                                                                        & -                                                                              & -                                                                              & 0.1807                                                                         & -                                                                                     \\ \midrule
\multicolumn{1}{c|}{\multirow{3}{*}{C}}          & \begin{tabular}[c]{@{}c@{}}Gaussian \\ (0.05)\end{tabular}    & 0.1078                                                                        & 0.1066                                                                        & 0.2405                                                                         & 0.1839                                                                         & 0.0650                                                                         & 0.1839                                                                                \\
\multicolumn{1}{c|}{}                            & \begin{tabular}[c]{@{}c@{}}Gaussian \\ (0.005)\end{tabular}   & 0.1304                                                                        & 0.0222                                                                        & -                                                                              & -                                                                              & 0.0222                                                                         & -                                                                                     \\
\multicolumn{1}{c|}{}                            & TBC                                                           & 0.1744                                                                        & 0.0328                                                                        & -                                                                              & -                                                                              & 0.0196                                                                         & -                                                                                     \\ \midrule
\multicolumn{1}{c|}{\multirow{3}{*}{\textbf{D}}} & \begin{tabular}[c]{@{}c@{}}Gaussian \\ (0.05)\end{tabular}    & 0.3661                                                                        & 0.0592                                                                        & -                                                                              & 0.1872                                                                         & 0.0592                                                                         & 0.1872                                                                                \\
\multicolumn{1}{c|}{}                            & \begin{tabular}[c]{@{}c@{}}Gaussian \\ (0.005)\end{tabular}   & ND                                                                            & ND                                                                            & ND                                                                             & ND                                                                             & ND                                                                             & ND                                                                                    \\
\multicolumn{1}{c|}{}                            & TBC                                                           & 0.3780                                                                        & 0                                                                             & -                                                                              & 0.2665                                                                         & 0                                                                              & 0.2637                                                                                \\ \midrule
\multicolumn{1}{c|}{\multirow{3}{*}{E}}          & \begin{tabular}[c]{@{}c@{}}Gaussian \\ (0.05)\end{tabular}    & 0.6653                                                                        & 0.1439                                                                        & -                                                                              & 0.1675                                                                         & 0.1439                                                                         & 0.1675                                                                                \\
\multicolumn{1}{c|}{}                            & \begin{tabular}[c]{@{}c@{}}Gaussian \\ (0.005)\end{tabular}   & 0.5825                                                                        & 0.1211                                                                        & -                                                                              & -                                                                              & 0.1211                                                                         & -                                                                                     \\
\multicolumn{1}{c|}{}                            & TBC                                                           & 0.5476                                                                        & 0.0133                                                                        & -                                                                              & -                                                                              & 0.0133                                                                         & -                                                                                     \\ \cmidrule(l){2-8} 
\end{tabular}
\caption{Band gap from HSE06 calculations with gaussian smearing and smearing width  $sigma$ equal to 0.05 and 0.005, and the tetrahedron method (TBC). "-"  mean unchanged values, "ND" means not done.}
\end{table}
\end{landscape}
\newpage

From table 7.6 we observe that the case of defect states is only a concern at larger smearing widths with Gaussian smearing. Contrary to previous cases, we find here finite band gaps despite of defect states. By comparing $E_G ^\text{up}$ and $E_G ^\text{dw}$ at $occ = 0.5$ and $occ = 0.01$, the defects appear to have a lesser role in spin up, as par SQS C the band gap in spin up is either consistent or only marginally different between the defect band gap and the hypothetical defect less band gap. $E_G ^\text{dw}$ on the other hand increase significantly by removing the defect states. The Gaussian smearing method is generally in better agreement with TBC at lower smearing width. But even in this case we find several dissimilarities. In A and E $E_G ^\text{dw}$ is larger with the Gaussian method, additionally $E_G ^\text{up}$ is much lower in A. In this project we have based our choice of numerical smearing on the advice on the VASP manual that state that for accurate total energies and density of states in semiconductors one should opt for the tetrahedron method \cite{ismear}.  However since our system is comprised of metals we include the results from utilizing Gaussian smearing as well. There are of course many more factors that affect the accuracy and reliability of both methods, but these are outside the scope of this project.

The fact that all 3 functionals and five SQS in majority agree on the presence of a band gap is in itself an overwhelmingly positive result that allow us to state with high certainty that the potential high-entropy silicide \ch{(CrFeMnNi)Si2} is in fact a semiconductor or possibly a half-metal based on the observed spin polarization and the utmost stable SQS. Regarding the 3 functionals applied in this project, we experience best cohesion between PBE and HSE06 that both agree on a spin up polarization of the band gap, while SCAN predicts more symmetric band gaps. This can also be seen from the magnetic moment, with PBE and HSE06 the final magnetic moment (per atom) is 0.083 $\mu_B$ across all SQSs, with SCAN this is reduced to half as much. This result could be related to the known drawbacks of SCAN and magnetic materials. In the nonmagnetic $\beta-$ \ch{FeSi2} structure we find better agreement between PBE and SCAN. Both correctly predict that the material is nonmagnetic, however compared to the experimental value of about 0.85 eV and the PBE band gap of 0.65 eV, we get a smaller band gap of 0.61 eV with SCAN. Thus the SCAN functional does not necessarily result in increased accuracy over PBE even in the nonmagnetic material. To conclude this section on the band gap \ch{(CrFeMnNi)Si2}, when studying the band gap with DFT, particularly PBE is well known to underestimate the band gap of the real material as in \ch{FeSi2}. Therefore a band gap found with PBE would with high probability be replicated/increased in the real material. 
 
\newpage 
\subsection{Pair distribution functions}
The probability distribution functions of SQS D and E can be seen bellow in figure 7.10, the PDFs corresponding to the remaining SQSs can be found in appendix .. . We include the PDFs of SQS D and B because as stated D is the most stable atomic configuration and hence the most representative of a potential real compound, and B to investigate distinctions between the half-metallic structure D and the semiconducting B which with HSE06 yielded substantial band gaps in both spins, recalling also that this is just very slightly bellow D in terms of stability. In the analysis we will put special emphasis on the nearest neighbor interactions since these are the most crucial in deciding the functional properties of a material. 
 
\begin{figure}[H]
	\centering
	\begin{subfigure}{\textwidth}
		\includegraphics[width=\textwidth]{results/fesi2/D_PDF2.png}
	\end{subfigure}
	\begin{subfigure}{\textwidth}
		\includegraphics[width=\textwidth]{results/fesi2/B_PDF.png}
	\end{subfigure}
	\caption{Probability distribution function of SQS D (top) and B (bottom)}
\end{figure}

We see that the relative positions of the PDFs remain consistent though both SQSs. With the aid of the ICSD (insert citation), we can compare the figure .. to the expected PDFs based on a number of experiments from a host of different compounds. As our compound contain a total of 15 different bonds, comparing each one to the ICSD values would be an exhaustive process. For our purpose we are satisfied by comparing the 4 different metal-Si bonds. We find that the preferred bond-length of TM-Si is observed at two values, the most dominant being the shorter of the two. For Fe-Si these are between 2.25-2.75 and 4-5, Mn-Si 2.25-2.75 and 3.5-5. Ni-Si lie between 2.25-2.5 and 3.85-5 and Cr-Si between 2.35-2.65 and 4-5.Clearly, the PDFs of the alloys are in good agreement with the listed values for Tm-Si bonds, with the most occurring bond length falling at around 2.4 Å for all TMs, and lesser occurrence between 4.0 - 4.5 Å. The height of the respective peaks is somewhat consistent in both structures, other than slightly reduced Fe-Si occurrence at 2.4 Å in B.

In contrast to the TM-Si bonds, we observe several distinctions between metal bonds in SQS D and B. Covering all would be tedious and not to insightful, instead we emphasize the bonds of Mn and Cr as this is where we found the biggest discrepancy in the PDOS. From the different TM-TM bonds (middle) of figure 8.8 we observe that the Mn-Fe bonds are most occurring at short distances in D and bigger distances in B, meaning that manganese and iron atoms are placed further from each other in structure D. \textbf{correct?} Similarly the bonds between Cr and Fe   indicate that these atoms lie closer in B than D. In contrast the nickel and manganese/chromium bonds point to a closer distance in B for Ni-Mn and Ni-Cr in D, and a greater distance between Ni and Mn in D and Ni and Cr in B. \textbf{Litt kronglete kanskje?} In terms of the homogeneous bonds, the properties of both Cr-Cr bonds and Mn-Mn bonds are more or less alike in both structures besides some majority at shorter distance in D (The red Cr-Cr line at 3Å is underneath the grey Ni-Ni line in B in figure 8.8 (bottom right)). A more significant distinction is that both Ni-Ni and Fe-Fe bonds are found at 3 Å and 4 Å in B, but exclusively 4 Å in D.    

Both the Fe-Fe and Ni-Ni bonds are in better agreement with the ICSD histograms, as the most occurring distance for these bonds are between 4-4.9 Å and additionally around 2.5 Å. \textbf{More comparisons to ICSD, ask O.M}. As a conclusion on the PDFs of this compound, we locate a pattern where the Si-Si bonds are identical and only very minor differences between TM-Si bonds in SQS D and B. This is a result of how the structures are generated with the SQS method. In th FeSi2 structure the silicon atoms are placed as before in the new supercells, but the TM elements are "randomly" distributed. Thus, it's reasonable that also here we would find the major differences between SQSs in the PDFs. 

\newpage
\subsection{SQS size}
Above we have presented the results of a high-entropy silicide \ch{(CrFeMnNi)Si2} investigated by 5 48 atom SQSs with a volume of $700$\r{A}$^3$. This intermediate size allowed for the use of more complex XC-functionals, and secondly enabled a broad study of distinct permutations and compositions as we will discover in the next chapters. However the application of the special quasi-random structures method to HEAs is not necessarily straightforward. Recalling from section 4.3 the first initial concern is the size of the SQS model and if it's sufficient enough to correctly model the disordered multi-component structure. In this section we will consider this problem by studying the difference between the 48 atom SQS to that of a 96 and 192 atom SQS with volume $1200$\r{Å}$^3$ and $2400$ \r{Å}$^3$ respectively. The computational demand for the 3 sizes is seen below in figure 7.11 where we plot the number of CPU hours as a function of number of atoms, which yield roughly a $N^3$ dependence. 

\begin{figure}[H]
\centering
\includegraphics[width=.7\textwidth]{results/SQS_time.png}
\caption{CPU time, \textbf{Make log plot instead}}
\end{figure}

Bellow we list the mean and standard deviation of the total energy, magnetic moment and enthalpy of formation of the 3 sizes in table 7.6, and band gaps in table 7.7. 

\begin{table}[H]
\centering
\begin{tabular}{@{}cccccc@{}}
\toprule
SQS size  & \multicolumn{2}{c}{\begin{tabular}[c]{@{}c@{}}Toten\\ (eV)\end{tabular}} & \multicolumn{2}{c}{\begin{tabular}[c]{@{}c@{}}Mag\\ ($\mu_B$)\end{tabular}} & \begin{tabular}[c]{@{}c@{}}$\Delta H$\\ (eV)\end{tabular} \\ \midrule
          & mean                                 & std                               & mean                                 & std                                  & mean                                                      \\ \midrule
48 atoms  & - 6.6105                             & ..                                & 0.0833                               & 0.0000                               & -11.5000                                                  \\
96 atoms  & - 6.6092                             & 0.0021                            & 0.0708                               & 0.0114                               & - 22.8752                                                 \\
192 atoms & - 6.6123                             & 0.0022                            & 0.0761                               & 0.0171                               & - 46.6654                                                 \\ \bottomrule
\end{tabular}
\caption{Overivew 48, 96 and 192 SQSs. }
\end{table}

As seen from table 7.4 both the total energy and magnetism remain more or less consistent throughout all sizes, this is a good indication of that the 48 atom model can adequately model the alloy. \textbf{Something on the formation enthalpy.} The band gap as seen in table 7.7 is first of all evident across all 3 SQS models and show similar polarization favoring the spin up direction. In several cases, we find that the magnitude of the band gap lessen with increasing SQS size. 

\begin{table}[H]
\centering
\begin{tabular}{@{}ccccc@{}}
\toprule
SQS size                                   & SQS        & \begin{tabular}[c]{@{}c@{}}$E_\text{G} ^{up, eigen}(0.5)$ \\ (eV)\end{tabular} & \begin{tabular}[c]{@{}c@{}}$E_\text{G} ^{dw, eigen}(0.5)$ \\ (eV)\end{tabular} & \begin{tabular}[c]{@{}c@{}}$E_\text{G} ^{tot, eigen}(0.5)$\\  (eV)\end{tabular} \\ \midrule
\multicolumn{1}{c|}{\multirow{5}{*}{48 atoms}}  & A          & 0.0815                                                                         & 0.0521                                                                         & 0.0281                                                                          \\
\multicolumn{1}{c|}{}                     & B          & 0.2932                                                                         & 0.0523                                                                         & 0.0523                                                                          \\
\multicolumn{1}{c|}{}                     & C          & 0.2355                                                                         & 0.0343                                                                         & 0.0343                                                                          \\
\multicolumn{1}{c|}{}                     & \textbf{D} & 0.3386                                                                         & 0                                                                              & 0                                                                               \\
\multicolumn{1}{c|}{}                     & E          & 0.3078                                                                         & 0.0495                                                                         & 0.0495                                                                          \\ \midrule
\multicolumn{1}{c|}{\multirow{5}{*}{96 atoms}}  & \textbf{A} & 0.1705                                                                         & 0.0442                                                                         & 0.0367                                                                          \\
\multicolumn{1}{c|}{}                     & B          & 0.1386                                                                         & 0.0270                                                                         & 0.0270                                                                          \\
\multicolumn{1}{c|}{}                     & C          & 0.1347                                                                         & 0.0363                                                                         & 0.0075                                                                          \\
\multicolumn{1}{c|}{}                     & D          & 0.0892                                                                         & 0.0398                                                                         & 0.0398                                                                          \\
\multicolumn{1}{c|}{}                     & E          & 0.1610                                                                         & 0                                                                              & 0                                                                               \\ \midrule
\multicolumn{1}{c|}{\multirow{5}{*}{192 atoms}} & A          & 0.1197                                                                         & 0.0321                                                                         & 0.0321                                                                          \\
\multicolumn{1}{c|}{}                     & B          & 0.1444                                                                         & 0                                                                              & 0                                                                               \\
\multicolumn{1}{c|}{}                     & C          & 0.1867                                                                         & 0                                                                              & 0                                                                               \\
\multicolumn{1}{c|}{}                     & D          & \textit{0.0478}                                                                & \textit{0.0339}                                                                         & 0                                                                               \\
\multicolumn{1}{c|}{}                     & \textbf{E} & \textit{0.0131}                                                                & \textit{0.0184}                                                                         & \textit{0.0131}                                                                          \\ \bottomrule 
\end{tabular}
\caption{Band gap of SQSs of 48, 96 and 192 atoms each of \ch{(CrFeMnNi)Si2}. The names are arbitrary, ie A in 48 does not equal A in 96 or 192.}
\end{table}

Similar to structure D in the 48 atom SQS we find that the 0 value in SQS E in the 48 atom model suffers from defect states and find $E_\text{G} ^{dw, eigen}(0.90, 0.10) = 0.016$ eV. The same is true for SQS B and C (192), but require $occ = 0.999, 0.001$ to locate a small nonzero spin down band gap. The band gap in SQS D and E (192) on the other hand is finite at $occ = 0.5$ but can be enlarged from increasing $occ$. In D we get $E_\text{G} ^{up, eigen}(0.99) = 0.075$ eV and $E_\text{G} ^{dw, eigen}(0.01) = 0.05$ eV and similarly $E_\text{G} ^{up, eigen}(0.99) = 0.05$ eV, and $E_\text{G} ^{dw, eigen}(0.01) = 0.048$ eV in E. In such cases where the eigenvalues inclusive of defect states return a finite band gap, the density of states does not. This is seen in figure .. for SQS E (192). 

\begin{figure}
\centering
\includegraphics[width=.8\textwidth]{results/fesi2/192_E_DOS.png}
\caption{Density of states of SQS E 192 atom SQS.}
\end{figure}    
 
Drawing any conclusion on the band gaps is difficult seeing as we find very different results within the all 3 sizes. Based solely on the most stable SQS it's clear that the larger cell produce a much lower and different band gaps compared to the moderate SQS sizes which are much more similar. However also here we find that the gap in the 96 cell is only abort half of the 48 atom cell. However as seen in table 7.7 we also find evidence of large band gaps in the larger cells in other atomic configurations. This goes back to section .. when we mentioned that one of the biggest drawbacks of the special quasi-random structures method is the large number of possible atomic configurations, thus in order to conclude between results in this project the most sensible point is to consider the most stable SQS, but as seen from the very varying properties between SQSs of the same model, this does not necaccarly been the most stable SQS if we trialed 20 SQSs instead of just 5. An additional point is the magnetic property, here we only applied one configuration to base the stabiliy on, thus it's very probable that fine tuning the magnetic moments could result in different properties.  

Looking at the pair distribution functions in figure 7.11 we see that the local ordering and short-range interactions is well represented and identical across all three sizes. The distinctions of preferences could as stated above simply be a product of the uniqueness of the SQSs more so than the size. On the other hand the larger SQSs clearly provide a better description of large-range interactions, that is not nearly as present in the smaller cell. However as seen in table 7.4 and in accordance with the fundamental concept of the special quasi-random structures method is that the functional properties is mostly determined by short-range effects in the lattice. Therefore, even though the bigger SQSs is a more accurate model the improvement is not justified from the cost, as illustrated in figure 7.12. And the apparent larger concern from appling the SQS method in this project is the uniqueness of each SQS.

\begin{figure}[H]
\begin{subfigure}{\textwidth}
\includegraphics[width=\textwidth]{results/PDF48.png}
\end{subfigure}
\begin{subfigure}{\textwidth}
\includegraphics[width=\textwidth]{results/PDF96.png}
\end{subfigure}
\begin{subfigure}{\textwidth}
\includegraphics[width=\textwidth]{results/PDF192.png}
\end{subfigure}
\caption{Pair distribution functions of SQS sizes (top) 48 atoms, (middle) 96 atoms, (bottom) 192 atoms}
\end{figure}
