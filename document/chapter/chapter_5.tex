\chapter{Practical application of DFT}
\label{sec:Practical DFT}

In this section we will present how the density functional theory discussed above can be applied and implemented in a computationally feasible manner to model various materials.

\textbf{Insert section in XC functionals here}.
With the echange-correlation functionals presented above, we now have everything in order to perform DFT calculations. To begin solving eq .., we need the single-electron wave-function, for a free electron this is a plane wave $\psi_k = Ae^{i\boldsymbol{k}\boldsymbol{r}}$. In a solid however, there exist a nonzero periodic potential $V(\boldsymbol{r}) = V(\boldsymbol{r} + \boldsymbol{R})$, the solution to the Shr\"{o}dinger equation is given by Bloch's theorem wich states that the solution takes the form
\begin{equation}
\psi_{\boldsymbol{k}}(\boldsymbol{r}) = u_{\boldsymbol{k}}(\boldsymbol{r})e^{i\boldsymbol{k}\boldsymbol{r}},    
\end{equation}
where $u_{\boldsymbol{k}}(\boldsymbol{r}$ is a bloch wave with identical periodicity to the supercell. And $\boldsymbol{k}$ is the wavevector. Along with eq(above), problems in DFT are solved in k-space or reciprocal space for convience sake. For instance a great deal of DFT calculations revolve around solving the integral 
\begin{equation}
    g = \frac{V_{\text{cell}}}{(2\pi)^3} \int_{\text{BZ}} g(\boldsymbol{k})d\boldsymbol{k},
\end{equation}
with BZ denoting that the integral be evaluated for all $\boldsymbol{k}$ in the Brillouin zone. This integral can be approximated by evaluating the integral at a set of discrete points and summing over the points with appropriatly assigined weigts. A larger set of points leads to more exact approximations. This method is called Legendre quadrature. The method for selceting these points in reciprocal space was devolped by Monkhorst and Pack in 1976, and simply put requieres a amount of kpoints in each direction in reciprocal space, in the form $N x N x N$. Recalling that reciprocal space is inverse to regular space, supercells with equal and large dimensions converge at smaller values of N, and inversly for cells of small dimsion. In supercells with different length axis, such as hexagonal cells, we use the notation $N x N x M$, where $M$ relate to the distincntly different axis. The amount of kpoints required can be fruther reduced by utulizing the symmetry of the cell, in which we can exactly approximate the entire BZ by extending a lesser zone through symmertry. This reduced zone is appropriartly named the irreducible Brillouin zone (IBZ). 

Metals in particular requiere a large set of kpoints to acchive accurate results. This is becouse we encounter discontinuies functions in the Brillouin zone around the fermi sufrace where the states discontinusly change from occupied to non-occupied. To reduce the cost of this operatin, there are two primary methods, tetrhaedon and smearing. The idea behind the tetrahedon method is to use the discrete set of k-points to fill the reciprocal space with tethraeda and interpolate the function within each tethraeda such that the function can be integrated in the entire space rather than at discrete points. The latter approach for solving discontinuos integrals is to smear out the discontinuity and thus transforming the integral to a continous one. A good analogy to this method is the fermi-dirac function, in which a small variable $\sigma$ transform a step-functino into a continious function that can be integrated by standard methods.

In addition to the number of kpoints, there is one more distinct parameter that must be specified in DFT calculations, namely the energy cutoff, or $E_{\text{cut}}$. This parameters arise from the Bloch function described previosly. In which $u_{\boldsymbol{k}}(\boldsymbol{r})$ was a bloch wave with the same periodicity as the supercell. This implies that the wave can be expanded by a set of special plane waves as
\begin{equation}
    u_{\boldsymbol{k}}(\boldsymbol{r}) = \sum_{\boldsymbol{G}} c_{\boldsymbol{G}}e^{i\boldsymbol{G}\boldsymbol{r}},
\end{equation}
where $\boldsymbol{G}$ is the reciprocal lattice vector. Combining this with eq ..(first eq for blcoh function) we get 
\begin{equation}
    \psi_{\boldsymbol{k}}(\boldsymbol{r}) = \sum_{\boldsymbol{G}} c_{\boldsymbol{k} + \boldsymbol{G}}e^{i(\boldsymbol{k} + \boldsymbol{G})\boldsymbol{r}}
\end{equation}
The consequense from this expression is that evaluating the wavefunction of an electron at a single $k$ point demand a summation over the entirity of reciprocal space. In order to reduce this computational burden, we can introduce a maximum paramater $E_{\text{cut}}$ to cap the calculations. This is possible becouse eq ..(above) is the solution of the Shr\"{o}dinger equation with kinetic energy 
\begin{equation}
    E = \frac{\hbar^2}{2m}|\boldsymbol{k} + \boldsymbol{G}|^2.
\end{equation}
Seeing as the solution with lower energies are the most interesting, we can limit the calculations of eq ..(2 above) to solutions with energy less than $E_{\text{cut}}$ given bellow
\begin{equation}
    E_{\text{cut}} = \frac{\hbar^2}{2m}G_{\text{cut}}.
\end{equation}
Thus, we can reduce the infinitly large sum above to a much more feasable calculation in 
\begin{equation}
    \psi_{\boldsymbol{k}}(\boldsymbol{r}) = \sum_{|\boldsymbol{k} + \boldsymbol{G}| < G_{\text{cut}}} c_{\boldsymbol{k} + \boldsymbol{G}}e^{i(\boldsymbol{k} + \boldsymbol{G})\boldsymbol{r}}
\end{equation}

\textbf{A summary on kpoints and ENCUT, plus a discussion on nummerical convergence and how to select kpoints and ENCUT}


A final consideration to how DFT is applied in practise is how the core electrons are handled. Tightly bound core electrons as opposed to valene electrons demand a greater number of plane-waves to converge. The most efficient method of reducing the expenses of core-electrons are so-called pseudopotentials. This method works by approximating the electron density of the core elecrons by a constant density that mimic the properties of true ion core and core electrons. This density is then remained constant for all subsequent calculations, ie only considering the valence electrons while regarding the core electrons as frozen-in. There are currently two popular types of psudopotentials used in DFT, so-called ultrasoft psudopotentials (USPPs) devoplped by Vanderbilt, and the projecter augmented-wave (PAW) method by Bloch. This project will exclusivly apply the latter. 

\textbf{Self-consistent iteration and relaxation discussion and figure}
