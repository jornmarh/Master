\chapter{Computational details}
\label{sec:Computation}

This section is intended to provide the nescssary details for reproduction of results to be presented later on. First we begin by describing the software used for the project. 

\section{Vienna Ab initio Simulation Package}
This software, often referred to as VASP is a package for ab initio quantom mechanics calculations using the projected augmented wave method and plane wave basis set. The intended methodolgy is DFT, but have been extended for methods post the original DFT-formulation. Calculations with VASP was carried out on the supercomputer fram, with a quantity of allocated time and resources provided på Uninett Sigma2,\textbf{add reference!}.

The strucutre of VASP rely on a set of input files and output files from the calculation, the input files required to perform a DFT computation in VASP are the following:
\begin{itemize}
    \item INCAR - this file provide the tags responsible for different methods, algorithms, parameters etc.
    \item POSCAR - this file is related to the crystal strucure of the system
    \item POTCAR - What psudopotential that is used
    \item KPOINTS - A file containing information on what KPOINTS will be used
    \item jobfile - This file contains information for the supercomputer regarding resources and such.
\end{itemize}
The capitalization displayed above is directly related to the requierements of the filesystem in the VASP/fram collaboration. Some important output files are:
\begin{itemize}
    \item CONTCAR - The relaxed crystal strucure after finalized calculation
    \item CHGCAR - This file contains the electron density after calculation
    \item EIGENVAL - Contains the solutions to the Kohn-Sham eigenfunctions
    \item DOSCAR - Infomration on the Density of States
    \item OUTCAR - Contains a list of all other information.
\end{itemize}
In this project, we began the calculation of every induvidual structure by testing the convergence of total energy with repsect to the number of k-points and cutoff energy. In VASP, the latter can be specified by setting the tag "ENCUT" in the INCAR file, we found 300 eV to yield producive results in terms of convergence and computation time for total energy calcuations, and 400 for ionic+volume relaxations. Regarding the number of points in the reciprical space, we carried out a great deal of simulations on numerous strucutes with distinct crystal strucutres and corresponding supercells, for this reason we employed a number of different sets of k-points depending on the strucure. Typically the number of points ranged from a 2x2x2 mesh to 4x4x4 mesh. With the smaller being required for hybrid functionals to converge. 

Upon realizing the convergence parameters, the strucures were allowed to relax both the ionic posistions, and cell volume with the quas-newton method and a convergence criterion of $1E-2$ for the forces and $1E-5$ for the total energy. However, the symmetry of the stucutre was forced constant by the use of vasp-std-noshear. This process was repeated two times before perfoming one final total energy calculation with GGA functional, and one with the SCAN functional, increasing the convergence criterion to $1E-6$. In certain cases, we also attempted to perform calculations with the hybrid functional HSE06.

The specific tags, algorithms, parameters and options of VASP that was in use throughout this project can be found at our github adress, but in particlar we would like to cover specificlly two factors/tags. Firstly is the tag ISPIN that correspond to the magnetism of the system. Considering this study revolved heavily around magnetic elements such as iron and Nickel, we used ISPIN=2 which allow for colinear spin-polarized calculations. However, there are many more magnetic orientations the system can adopt besided colinear, therfore the final total energies we found may not be the true lowest energies. But given the allocated duration and resources of this project, this is a understood consequence. Secondly is the type of smearing that was used for the different calculations. The preferred method for accurate total energies and density of states in semiconductors is the tethraedon method, and for accurate forces in metals the Methfessel-Paxton method is recommended. However, our system contains both metals and a large portion of Si. For this reason we used a combination of smearing methods. For the relaxation and minimzation of forces, we used gaussian smearing with smearing width $\sigma = 0.05$, as this method provide accurate foces in both metallic and semicondicing materials. And to claculate the total energy and DOS, we used the tetradon method, as recommended. One interesting result of this project, is that we were not able to converge calculations with hybrid functinoals using the tetrahedon method and was thus forced to adopt the gaussion method in this case additionaly. 

\textbf{Band structure/DOS and band-unfolding?}

\section{Generation of SQS}
The generation of special quasi-random structures as described in section .., was done by utilizing the Temperature Dependent Effective Potential (TDEP) method. This package, devolped by Olle Hellman, offers a wide range of tools primarly intended for studies of finite temperature lattice dynamics. In this project we utilize the program generate-strucutre within the TDEP package to contruct SQS's. The work of TDEP is the result of an unpuplished PHD thesis by Nina Shulumba \textbf{(Insert citation)}, thus the documentation on the software and generate-structure script is limited, please refer to the original author for more information. 

In this project, we constructed SQS's by first transforming the cif-files of a given initial structure, for instance that of $FeSi_2$, to a primitive unit cell. The SQS's was genereated by the same principles explained in section .., for each strucure we created 5 distinct SQS's of an equal size under the constraint that the 3d atoms be distributed eqvimolar in the systen. Precise file formats and such can be found at github. Another approach could have been to construct SQS's of specific cell counts instead of total number of atoms, however this quikly lead to extremly large supercells, up to 256 atoms, that simply would not converge to our best efforts. 
\textbf{Insert figure of SQS's before relaxation, for some of the strucures and give a brief summary.}

\section{Utility scripts}
During the course of the projects lifetime, several shell and python scripts was devolped by myself and/or provided to me by my superviosor Ole Martin Løvvik and his team of researchers at Sintef. These can be located at the github adress :...