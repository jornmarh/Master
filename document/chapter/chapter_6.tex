\chapter{Computational details}
\label{sec:Computation}

\section{Settings and dependencies}

The computations were performed on resources provided by Sigma2 - the National Infrastructure for High Performance Computing and Data Storage in Norway, utilizing the Vienna Ab initio Simulation Package (VASP) \cite{vasp1}, \cite{vasp2}, \cite{vasp3}, \cite{vasp4}. As discussed in chapter 5, we employ the projector-augmented-wave method and PBE GGA, in addition to SCAN and HSE06 in certain instances. For the structures studied in this project, we found an energy cutoff of 300 eV and 400 eV suitable for electronic and geometric relaxations respectively. In regards to the number of k-points, we used a gamma centered mesh with a density of 4 per $\AA^{-1}$. The convergence of these parameters with respect to the total energy can be seen in table 6.1. The convergence tests were conducted for a 48-atom supercell of a \ch{Cr4Fe4Mn4Ni4Si32} alloy, that is representative for all alloys studied in this project. From table 6.1 we see that the cutoff energy could be increased further for optimal convergence, however to spare computational resources we opt for the lower value. Nevertheless, between 300 eV and 350 eV, the total energy only alters by 3 meV, which is around the limits of computational accuracy of VASP calculations.  

The geometric relaxation of ionic positions and cell volume was carried out in two subsequent runs with convergence criterion of \num{1E-2} $\si{\eV}/\AA$ for the forces and \num{1E-5} eV for the total energy, with Gaussian smearing (ISMEAR = 0 in VASP) and smearing width $\sigma$ equal to 0.05 eV. After successful geometric relaxation, the structures underwent a final electronic relaxation with the tetrahedron method with Bl\"{o}ch corrections (TBC, ISMEAR = -5 in VASP), and energy criterion of \num{1E-6} eV between consecutive Kohn-Sham iterations. Calculations with the HSE06 functional in many instances proved difficult to converge electronically. This was solved by two measures: firstly we reduced the density of k-points from 4 $\AA^{-1}$ to 2 per $\AA^{-1}$. Secondly, we found that HSE06 computations converged much quicker with Gaussian smearing compared to the tetrahedron method. Thus, in order to successfully and economically carry out calculations with HSE06 and the tetrahedron method, we first calculated the charge density with Gaussian smearing and reapplied the calculated density in a subsequent HSE06 calculation with the tetrahedron method. Magnetic materials consisting of ferromagnetic elements such as iron, nickel and cobalt, were handled with the setting ISPIN = 2 and the default value of MAGMOM = NIONS * 1.0 in VASP, which specifies the initial magnetic moment for each atom.  Further customization and testing of magnetic orderings such as antiferromagnetic, ferrimagnetic and ferromagnetic were considered beyond the scope of this project.  

The SQS method was implemented through the $\text{generate}-\text{structure}$ script in the Temperature dependent effective potential (TDEP) package \cite{tdep}, developed by Hellman and Shulumba. The CIF-file of $\beta$-\ch{FeSi2} that we used to build are alloys were obtained from Materials Project \cite{Jain2013}. Furthermore, Materials Project was used to compare and measure the validity of calculations. To extract and post-process the VASP calculated data, we relied on both VASPKIT \cite{vaspkit} and pymatgen \cite{pymatgen}.  

\begin{table}[H]
\centering
\begin{tabular}{@{}cc@{}}
\toprule
\begin{tabular}[c]{@{}c@{}}Cutoff energy\\ (eV)\end{tabular} & \begin{tabular}[c]{@{}c@{}}Total energy\\ (eV)\end{tabular} \\ \midrule
200                                                          & -6.444                                                      \\
250                                                          & -6.568                                                      \\
300                                                          & -6.586                                                      \\
350                                                          & -6.589                                                      \\
400                                                          & -6.588                                                      \\
450                                                          & -6.589                                                      \\ \bottomrule
\end{tabular}
\hspace{1cm}
\begin{tabular}{@{}cc@{}}
\toprule
\begin{tabular}[c]{@{}c@{}}K-points\\ (per $\AA^{-1}$)\end{tabular} & \begin{tabular}[c]{@{}c@{}}Total energy \\ (eV)\end{tabular} \\ \midrule
3                                                            & -6.584                                                       \\
4                                                            & -6.586                                                       \\
5                                                            & -6.586                                                       \\
6                                                            & -6.586                                                       \\ \bottomrule
\end{tabular}
\caption{Convergence tests of the cutoff energy and density of k-points, with respect to the total energy. The tests were conducted for a 48-atom supercell of \ch{Cr4Fe4Mn4Ni4Si32}.}
\end{table}


\section{Material}
In this project we have constructed high-entropy silicides based on the $\beta-$ \ch{FeSi2} compound. The unit cell of this material adopts the orthorhombic CMCE space group, and consists of 16 iron atoms and 32 silicon atoms. For each composition, we generate five distinct SQSs of equivalent geometry and composition, that only vary by the atomic configuration. We have emphasized a particular composition of the 3d elements Cr, Fe, Mn, and Ni in a \ch{Cr4Fe4Mn4Ni4Si32} alloy, where the 3d elements are distributed equimolarly over the Fe-sites in the $\beta-$ \ch{FeSi2} crystal structure. These SQSs can be seen in figure 6.1. In addition to the \ch{(CrFeMnNi)Si2} composition, we have tested alternative compositions with varying distributions of 3d elements and different elements. These supercells has been generated by an identical procedure and remain consistent with the 48 atom SQS model.

\begin{figure}[H]
\begin{subfigure}{0.5\textwidth}
\includegraphics[width=\textwidth]{method/sqs/A.jpg}
\caption{A}
\end{subfigure}
\hfill
\begin{subfigure}{0.5\textwidth}
\includegraphics[width=\textwidth]{method/sqs/B.jpg}
\caption{B}
\end{subfigure}
\begin{subfigure}{0.5\textwidth}
\includegraphics[width=\textwidth]{method/sqs/C.jpg}
\caption{C}
\end{subfigure}
\hfill
\begin{subfigure}{0.5\textwidth}
\includegraphics[width=\textwidth]{method/sqs/D.jpg}
\caption{D}
\end{subfigure}
\begin{subfigure}{0.5\textwidth}
\includegraphics[width=\textwidth]{method/sqs/E.jpg}
\caption{E}
\end{subfigure}
\caption{Five distinct 48-atom SQSs of \ch{Cr4Fe4Mn4Ni4Si32} based on the $\beta-$ \ch{FeSi2} crystal structure. Manganese atoms are represented as purple spheres, chromium as dark blue and silicon as light blue, followed by iron and nickel presented as gold and silver spheres respectively. The respective SQSs are denoted as A, B, C, D and E. Figures illustrated with VESTA \cite{vesta}}
\label{sqs_FeSi2}
\end{figure}